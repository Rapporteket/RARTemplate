% For print, bruk dokumentklassen 'book'. Bruk 'report' for å unngå blanke
% sider
\documentclass[norsk, a4paper, twocolumn]{report}

\usepackage[utf8x]{inputenc}
\usepackage[colorlinks=true]{hyperref}
\usepackage{babel}
\usepackage{authblk}
\usepackage{longtable}
\usepackage{multicol}
\usepackage{color}
\usepackage{wasysym}

% Ikke bruk innrykk for nye avsnitt, men heller litt større vertikal avatand
\setlength{\parindent}{0pt}
\setlength{\parskip}{1ex plus 0.5ex minus 0.2ex}

% Sett registerets navn
\def \registernavn {\textit{Navn på register}}

%% tittelting
\title{\registernavn \\ \textbf{Årsrapport \textit{[årstall]}\\
Plan for forbedringstiltak \textit{[årstall+1]}}}

\author[1]{Ola Normann}
\author[2]{Medel Svensson}
\author[1]{Kari Normann}
\affil[1]{et Sykehus, et Foretak, et Sted}
\affil[2]{en annen adresse, et annet Sted}

\renewcommand\Authands{ og }
\renewcommand\Authfont{\scshape}
\renewcommand\Affilfont{\itshape\small}

% stil for veiledende tekst
\definecolor{guidegray}{rgb}{0.2,0.2,0.2}
\newcommand{\guide}[1] {
	\textit{[\textcolor{guidegray}{#1}]}
	}
%% ---

% overstyre mulige ord-deling-er
\hyphenation{stadium-vurder-ing system-atiske pasi-ent-rap-porterte}


\begin{document}


\maketitle


\onecolumn
%\frontmatter



% Ta ut hele kapittelet ved bruk til faktisk rapport.
% Av en eller annen grunn ønskes også versjonsoversikten tatt ut fra malen\ldots
% Tas inn igjen ved å fjerne løkka rundt

%\iffalse

\chapter*{Om årsrapportmalen}
Etablering av en mal for årsrapport for anvendelse av de nasjonale medisinske
kvalitetsregistre gjøres på bestilling av den interregionale styringsgruppen
(styringsgruppen).
SKDE står for det praktiske arbeidet med malen, og innholdet er basert på
en rekke vedtak gjort i styringsgruppen samt innspill til justering fra
relevante aktører i registermiljøet.

Ved bruk av malen til etablering av faktiske årsrapporter skal informasjonen
som inngår i dette kapittelet fjernes.

Malen vil være et levende dokument som
forvaltes av SKDE som fortløpende innarbeider alle vedtatte endringer. Siste
versjon av malen vil således kunne fås ved henvendelse til SKDE. Dokumentet
kan fritt distribueres. For full utnyttelse av malen bør årsrapporten
produseres i \LaTeX. Til dette kan man eksempelvis bruke programvaren
\href{http://texstudio.sourceforge.net/}{TexStudio}\footnote{http://texstudio.sourceforge.net/}
som er fritt tilgjengelig og gratis i bruk.

\begin{table}[ht]
  \centering
  \begin{tabular}{lrp{8cm}p{2.5cm}}
    \hline
    Versjon & Dato & Aktivitet & Ansvarlig \\
    \hline
    0.1 & 24. januar 2013 & Opprettet første gang & Are Edvardsen \\
    0.2 & 1. februar 2013 & Endret etter innspill fra nodemøtet &
    Are Edvardsen \\
    0.3 & 14. mars 2013 & Endret etter innspill fra Leif Ivar Havelin,
    Svein Rotevatn,
    Reinhard Seifert, Sandra Julsen Hollung, Gro Andersen, Tore Solberg og
    Anne Marie Fenstad & Are Edvardsen \\
    0.4 & 2. april 2013 & Mindre endringer etter første møte i ekspertgruppen &
    Are Edvardsen \\
    0.5 & 7. mai 2013 & Etter interne innspill, restrukturering til 1)
    årsrapport og 2) planlagte tiltak. Mer veiledende tekst. Del ang.
    foretaksspesifikk rapportering er tatt ut av dokumentet & Are Edvardsen \\
    0.6 & 16. mai 2013 & Endring i beskrivelse av
    dekningsgrad, noen nye referanser og sammenslåing i generell
    registerbeskrivelse & Are Edvardsen \\
    0.7 & 4. juni 2013 & En hel del endinger og forenklinger etter diskusjon i
    ekspertgruppa 31. mai. Blant annet er sammendrag i tabellform med lenker
    til resten av dokumentet tatt ut & Eva Stensland, Are Edvardsen \\
    0.9 & 13. juni 2013 & Minimale endringer etter inspill fra møte i
    interregional styringsgruppe 12. juni & Eva Stensland \\
    0.91 & 18. juni 2013 & Nytt kapittel med generell bakgrunn og veiledning.
    Mindre endringer på form & Eva Stensland, Are Edvarsen \\
    1.0 & 18. juni 2013 & For bruk i årsrapporter for 2012 & Are Edvardsen \\
     & & & \\
    1.1 & 3. april 2014 & Endringer ift revidert stadieinndeling. Reversering
    av tidligere endringer: flytte Resultater frem og ta tilbake
    tabularisk sammendrag for hjelp til vurdering av stadium & Are Edvardsen,
    Eva Stensland \\
    1.2 & 4. april 2014 & Tabularisk sammendrag erstattet med stadiumvurdering
    i egen Del med referanser til øvrige deler av dokumentet & Are Edvardsen,
    Eva Stensland \\
    \hline
  \end{tabular}
  \caption{Endringslogg for dette dokumentet. Gjeldende versjon er siste
  oppføring i denne tabellen.}
  \label{tab:log}
\end{table}

%\fi

\chapter*{Bakgrunn og veiledning til utfylling}

\section*{Bakgrunn}
En årsrapport fra et medisinsk kvalitetsregister bør utarbeides først og
fremst for å vise hvilken nytte registeret har hatt for helsetjenesten, og
hvordan registeret kan brukes til forbedring. Årsrapporten bør utformes slik
at den også kan leses og forstås av personer utenfor det aktuelle fagmiljø.

Alle nasjonale medisinske kvalitetsregistre skal sende inn årsrapport til adm.
dir. ved driftsansvarlig RHF innen 15. oktober påfølgende år. Fra og med 2012
er dette et krav for tildeling av ressurser.

Denne malen for årsrapport er utarbeidet av Nasjonalt servicemiljø for
kvalitetsregistre på bestilling av interregional styringsgruppe, og ble
godkjent på møtet i interregional styringsgruppe 12. juni 2013 for bruk av alle
nasjonale medisinske kvalitetsregistre. Malen inneholder sentrale
rapporteringselementer som blant annet har sitt utgangspunkt i
\href{http://www.kvalitetsregistre.no/getfile.php/Norsk/Dokumenter/Stadieinndelingsmodell.pdf}{stadieinndelingssystemet}\footnote{http://www.kvalitetsregistre.no/getfile.php/Norsk/Dokumenter/Stadieinndelingsmodell.pdf}
for kvalitetsregistre, og en resultatdel.

For å kunne gi en samlet oversikt over nasjonale kvalitetsregistres
årsrapporter ber vi om at kopi av rapporten også sendes SKDE innen fristen
15. oktober. \href{http://www.kvalitetsregistre.no/aktuelt/ekspertgruppens-foerste-moete-article739-157.html}{Ekspertgruppen}\footnote{http://www.kvalitetsregistre.no/aktuelt/ekspertgruppens-foerste-moete-article739-157.html}
vil i november måned gjøre en gjennomgang av alle årsrapportene for inneværende
årsrapportperiode, og kategorisere de nasjonale kvalitetsregistrene i henhold
til stadieinndelingssystemet.

Det vil bli gjort en årlig evaluering og eventuelt revisjon av årsrapportmalen,
og innspill fra kvalitetsregistrene på erfaringer med innhold og funksjonalitet
er derfor ønskelig.


\section*{Veiledning til utfylling}

Kapittel \ref{cha:reg}-\ref{cha:pas} i malen er beskrivende, og utfylles så
langt det er mulig. Det vil
være mange registre som mangler informasjon for utfylling av ett eller flere
underkapitler. Ved manglende informasjon, fjerner man det aktuelle
underkapitlet. Det er laget en veiledende tekst til alle underkapitler som har
som hensikt å beskrive hvilken informasjon man ønsker fylt inn. I kapittel
\ref{cha:dek} og \ref{cha:dat}
er begrepet ”institusjon” brukt. Her fyller registeret inn informasjon på
foretaks-, sykehus- eller avdelingsnivå avhengig av hvilken informasjon som er
tilgjengelig i hvert enkelt register. 

Kapittel \ref{cha:res} er resultatdelen av årsrapporten, og her fyller det
enkelte
register inn de resultater (tabeller, figurer og tekst) de ønsker å formidle.
Det er et ønske om at man viser resultater fra de viktigste kvalitetsmål i
registeret, og at resultatene formidles på sykehus eller foretaksnivå dersom
mulig.

I hver helseregion finnes det en representant for det nasjonale servicemiljøet
for medisinske kvalitetsregistre som kan svare på spørsmål angående
årsrapporter. Kontaktinformasjon til disse finnes på
servicemiljøets \href{http://www.kvalitetsregistre.no/servicemiljoe/category158.html}{nettsider}\footnote{http://www.kvalitetsregistre.no/servicemiljoe/category158.html}.


%\mainmatter

\tableofcontents
\part{Årsrapport}\label{par:rap}

\chapter{Sammendrag}
\guide{Kortfattet sammendrag av de viktigste elementer man fra registerets side ønsker å formidle i årsrapporten.}

\section{Tabularisk oppsummering}
% for å se hvordan dette kan ta seg ut i tabellform, lag en fylltekst 	 
 \def \fyll {På den ene siden går det stadig oppover, men på den andre går 	 
 det lukt åt skogen. Ellers er det mest på det jevne og litt rundtomkring\ldots} 	 
  	 

\twocolumn




\chapter{Registerbeskrivelse}\label{cha:reg}
\guide{Informasjon til dette kapitlet hentes fra protokoll, søknad om nasjonal status, konsesjonssøknad etc.}


\section{Bakgrunn og formål}
\subsection{Bakgrunn for registeret}\label{sec:bak}
\subsection{Registerets formål}\label{sec:for}

\section{Juridisk hjemmelsgrunnlag}\label{cha:jur}

\section{Faglig ledelse og databehandlingsansvar}\label{cha:led}

\subsection{Aktivitet i styringsgruppe/referansegruppe}
\guide{Antall møter, viktige saker som er behandlet mv.}


\chapter{Metoder for fangst av data}
\guide{Beskriv hva/hvem som registrerer (eksempelvis sykehusenes egne
fagsystemer, helsearbeidere, pasienter) og hvordan innsamlingen skjer
(eksempelvis elektronisk eller på papirskjema).}




\chapter{Datakvalitet}\label{cha:kva}
\guide{Status og evaluering av registerets  datakvalitet.}

\section{Metoder for intern sikring av datakvalitet}\label{sec:sik}
\guide{Beskriv infrastruktur, rutiner og/eller punktaksjoner (eksempelvis
opplæring, informasjonsarbeid, tekniske støttesystemer) og tilbakekobling mellom register og innregistrerende enheter som ivaretar riktige og
komplette data.}

\section{Metode for validering av data i registeret}\label{sec:metval}
\guide{Beskriv hvilke eksterne kilder (registre) som er benyttet og hvordan
dette er sammenholdt med registerets data (design/subsampling).} 

\section{Vurdering av datakvalitet}\label{sec:valdat}
\guide{Angi viktige funn og en vurdering av resultatene fra
valideringsstudien slik som funn av systematiske feil, tilfeldige feil og
antatt effekt dette har på validitet for hele registeret.}




\chapter{Dekningsgrad}\label{cha:dek}
\guide{Dekningsgrad oppgis på institusjons- og individnivå. Dersom
et register ett år mottar registreringer fra tre
av totalt fire mulige institusjoner blir dekningsgraden på
institusjonsnivå 75\%. Dekningsgraden på individnivå angir
forholdet mellom registrerte individer og det man basert på uavhengige
kilder (f.eks.
\href{http://www.kvalitetsregistre.no/dekningsgradsanalyser/category358.html}
{NPR}\footnote{\url{http://www.kvalitetsregistre.no/dekningsgradsanalyser/category358.html}})
vet eller antar skal være det totale antall mulige registrerbare
individer i det aktuelle året.}

\section{Antall registreringer}\label{sec:reg}
\guide{Per 31. desember for rapporteringsåret, angi antall
individer/hendelser totalt og per institusjon. Kan også brytes opp i antall per år.}

\section{Metode for beregning av dekningsgrad}\label{sec:met}
\guide{Angi kilder/registre som er brukt, for hvilken periode beregningene
gjelder for og hvordan analysene er gjort (metode).}

\section{Dekningsgrad på institusjonsnivå}\label{sec:endek}
\guide{Angi resultatet av dekningsgradsanalysen på institusjonsnivå
(eksempelvis sykehus eller helseforetak),
eventuelt også endringer fra tidligere år.}

\section{Dekningsgrad på individnivå}\label{sec:obs}
\guide{Angi resultat av dekningsgradsanalyse på individnivå.}




\chapter{Fagutvikling og kvalitetsforbedring av tjenesten}\label{cha:fag}
\guide{Beskrivelse av registerets fagutvikling og kvalitetsforbedringstiltak
i rapporteringsperioden.}

\section{Registerets spesifikke kvalitetsmål}\label{sec:regspe}
\guide{Beskriv de mål/variabler/indikatorer som angir grad av kvalitet.
Se også
\href{http://www.kvalitetsregistre.no/getfile.php/Norsk/Dokumenter/Handbok_120411_siste.pdf}{Registerhåndboka}
\footnote{\url{http://www.kvalitetsregistre.no/getfile.php/Norsk/Dokumenter/Handbok_120411_siste.pdf}}
for ytterligere beskrivelse. I den grad registeret registrerer nasjonale
kvalitetsindikatorer, tas dette med.}





\section{Bidrag til utvikling av nasjonale retningslinjer og handlingsplaner/-program}\label{sec:retut}
\guide{I den grad det er relevant, beskriv hvordan registeret bidrar til
etablering av nasjonale retningslinjer, handlingsplaner/-program}

\section{Bruk av nasjonale retningslinjer}\label{sec:retbru}
\guide{I den grad de eksisterer, beskriv i hvor stor grad deltagende enheter
etterlever etablerte nasjonale retningslinjer.}


\section{Kvalitetsforbedrende tiltak initiert av registeret}\label{sec:brures}
\guide{Angi konkrete prosjekter
for kvalitetsforbedring initiert av registeret. Beskriv
hvordan registeret bidrar til oppstart og gjennomføring
av kvalitetsforbedringsarbeid lokalt hos innregistrerende institusjoner.}

\section{Evaluering av kvalitetsforbedrende tiltak (endret praksis)}\label{sec:evakva}
\guide{Beskriv resultater av gjennomførte tiltak for kvalitetsforbedringer
i kap. \ref{sec:brures}.}

\section{Komplikasjoner og uønskede hendelser}\label{sec:kom}
\guide{Beskriv hva som registreres av komplikasjoner og/eller uønskede
hendelser i forbindelse med den behandling som registeret omfatter.}




\chapter{Formidling av resultater}\label{cha:dat}
\guide{Status og evaluering av formidlingsform- og frekvens
av resultater fra registeret. Resultater i seg selv beskrives i kap.
\ref{cha:res}.}


\section{Resultater tilbake til deltagende fagmiljø}\label{sec:resfag}
\guide{Beskriv form og frekvens,  eventuelt tilbakemeldinger på relevans,
endringsønsker mm.}

\section{Resultater til administrasjon og ledelse}\label{sec:resled}
\guide{Beskriv form og frekvens, og hvordan denne
informasjonen formidles til ledelse og administrasjon.}

\section{Resultater til pasienter}\label{sec:respas}
\guide{Beskriv form og frekvens og eventuelt tilbakemeldinger fra pasienter,
særlig i forhold til nytten av informasjonen fra registeret.}

\section{Offentliggjøring av resultater på institusjonsnivå}\label{sec:off}
\guide{Beskriv hvor/til hvem, form og frekvens. Resultater i seg selv beskrives i kap. \ref{cha:res}.}






\chapter{Samarbeid og forskning}\label{cha:for}
\guide{Status og evaluering av samarbeid med andre registre/fagmiljø og  vitenskapelig aktivitet.}

\section{Samarbeid med andre helse- og kvalitetsregistre}\label{sec:samfag}
\guide{Beskriv eventuelle samarbeid registeret har med andre registre eller relevante fagmiljø, nasjonalt eller internasjonalt.}

\section{Vitenskapelige arbeider}\label{sec:vitarb}
\guide{Angi bidrag fra registeret i fagfellevurderte tidsskrifter,
vitenskapelige konferanser, utdanning av doktorgradsstipendiater eller
lignende.}



\chapter{Pasientrapportering, demografi og sosiale forhold}\label{cha:pas}
\guide{Status og evaluering av registerets bruk av pasientrapporterte
resultater samt
hvordan resultater fra registeret varierer med demografiske og sosiale forhold
i pasientpopulasjonen.}

\section{Pasientrapporterte resultatmål}\label{sec:pasutk}
\guide{Beskriv hvilke instrumenter som benyttes.}

\section{Demografiske ulikheter}\label{sec:demuli}
\guide{Angi hvilke variabler som er tilgjengelig i registeret (alder, kjønn,
bosted).}

\section{Sosiale ulikheter}\label{sec:sosuli}
\guide{Beskriv hvilke målinger som benyttes (sivilstatus, utdanning, inntekt,
etnisitet).}


\chapter{Resultater (tradisjonell årsrapport/-statistikk)}\label{cha:res}
\guide{Fyll inn resultater. Relativt uavhengig ift resten av dokumentet. Dette er ``Årsrapporten'' i tradisjonell forstand.}




\onecolumn

\part{Plan for forbedringstiltak}\label{par:for}
\chapter{Momentliste}
\guide{Basert på beskrevet status og evaluering av registeret, beskriv de
tiltak som er planlagt gjennomført fra og med neste rapporteringsperiode.
Punktlisten under kan brukes som et utgangspunkt og hjelp til hva som kan
inngå i en slik beskrivelse. For ytterligere forklaring av de ulike punktene,
se respektive kapitler under Del \ref{par:rap}: \nameref{par:rap}.}

\begin{itemize}
  \item Datafangst
    \begin{itemize}
      \item Forbedring av metoder for fangst av data
    \end{itemize}
  \item Datakvalitet
    \begin{itemize}
      \item Endringer av rutiner for intern kvalitetssikring av data
      \item Oppfølging av resultater fra validering mot eksterne kilder
    \end{itemize}

  \item Dekningsgrad
    \begin{itemize}
      \item Nye registrerende enheter/avdelinger
      \item Forbedring av dekningsgrad på individnivå i registeret
    \end{itemize}
  \item Fagutvikling og kvalitetsforbedring av tjenesten
    \begin{itemize}
      \item Nye kvalitetsmål
      \item Bidrag til etablering av nasjonale retningslinjer
      \item Registrerende enheters etterlevelse av nasjonale retningslinjer
      \item Økt bruk av resultater til kvalitetsforbedring i hver enkelt institusjon
      \item Prioriterte, faglige forbedringsområder
    \end{itemize}
  \item Formidling av resultater
    \begin{itemize}
      \item Forbedring av resultatformidling til deltagende fagmiljø
      \item Forbedring av resultatformidling til administrasjon og ledelse
      \item Forbedring av resultatformidling til pasienter
      \item Forbedring av hvordan resultater på institusjonsnivå offentliggjøres
    \end{itemize}

  \item Samarbeid og forskning
    \begin{itemize}
      \item Nye samarbeidspartnere
      \item Forskningsprosjekter og annen vitenskapelig aktivitet
    \end{itemize}
  \item Pasientrapportering, demografi og sosiale forhold
    \begin{itemize}
      \item Nye pasientrapporterte resultater som skal inn i registeret
      \item Utvidet bruk av pasientrapporterte resultater
      \item Nye demografiske variabler som skal inn i registeret
      \item Utvidet bruk av demografiske variabler
      \item Nye variabler som beskriver sosiale forhold hos pasientene
      \item Utvidet bruk av resultater som beskriver sosiale ulikheter
    \end{itemize}
\end{itemize}



%\listoffigures
%%\listoftables



\part{Stadiumvurdering}
\chapter{Referanser til vurdering av stadium}
\guide{Oversikt over vurderingspunkter som legges til grunn for
stadiuminndeling av registeret med referansert til relevant informasjon
gitt i årsrapporten. Denne delen skal ikke fylles ut, men er ment som en
hjelp til registeret og ekspertgruppen i vurdering av registerts
``modenhet''.}

\begin{longtable}{rp{10cm}lc}
  \caption[Vurderingspunkter for stadium \registernavn]
  {Vurderingspunkter for stadium \registernavn} \\
  \hline 	 
  Nr & Beskrivelse & Referanse & Ok \\ 	 
  \hline 	 
  \endfirsthead 	 
  \caption[]{forts.}\\ 	 
  \hline 	 
  Nr & Beskrivelse & Referanse & Ok \\
  \hline 	 
  \endhead
  \\
  \multicolumn{4}{c}{\textit{Tabellen fortsetter på neste side}} \\
  \hline
  \endfoot 	 
  \hline 	 
  \endlastfoot 	 	 
   & \textbf{Stadium 2} & & \\
  1 & Er i drift og samler data nasjonalt & \ref{cha:res} & \Square \\
  2 & Presenterer resultater på nasjonalt nivå & \ref{cha:res} & \Square\\
  3 & Har en konkret plan for gjennomføring av dekningsgradsanalyser
    & \ref{sec:met} & \Square\\
  4 & Har en konkret plan for gjennomføring av analyser og løpende
      rapportering av resultater på sykehusnivå tilbake til deltakende
      enheter & \ref{sec:resfag} & \Square\\
      5 & Har en oppdatert plan for videre utvikling
      & \ref{cha:res}, Del \ref{par:for} & \Square\\
   & & & \\

   & \textbf{Stadium 3} & & \\
  6 & Kan redegjøre for registerets datakvalitet
    & \ref{sec:sik}, \ref{sec:metval}, \ref{sec:valdat} & \Square\\
  7 & Har beregnet dekningsgrad mot uavhengig ekstern datakilde
    & \ref{sec:met}, \ref{sec:endek}, \ref{sec:obs} & \Square\\
  8 & Registrerende enheter har tilgang til egne og nasjonale resultater
    & \ref{sec:resfag} & \Square\\
  9 & Presenterer deltakende enheters etterlevelse av nasjonale retningslinjer
  & \ref{sec:retbru} & \Square\\
  10 & Har identifisert kliniske forbedringsområder basert på analyser fra
  registeret & \ref{sec:brures} & \Square\\
  11 & Brukes til klinisk kvalitetsforbedringsarbeid
    & \ref{sec:brures}, \ref{sec:evakva} & \Square\\
  12 & Resultater anvendes vitenskapelig & \ref{sec:vitarb} & \Square\\
  13 & Presenterer resultater for PROM og PREM & \ref{sec:pasutk} & \Square\\
  14 & Har en oppdatert plan for videre utvikling
    & \ref{cha:res}, Del \ref{par:for} & \Square\\
   & & & \\

   & \textbf{Stadium 4} & & \\
  15 & Kan dokumentere registerets datakvalitet gjennom en valideringsstudie
    & \ref{sec:valdat} & \Square\\
  16 & Presenterer oppdatert dekningsgradsanalyse hvert 2. år
  & \ref{sec:met}, \ref{sec:endek}, \ref{sec:obs} & \Square\\
  17 & Har dekningsgrad over 80\% & \ref{sec:obs} & \Square\\
  18 & Registrerende enheter har løpende tilgang til oppdaterte egne og
  nasjonale resultater & \ref{sec:resfag}, \ref{sec:off} & \Square\\
  19 & Bidrar til utvikling av nasjonale retningslinjer & \ref{sec:retut}
    & \Square\\
  20 & Presentere resultater på sosial ulikhet i helse & \ref{sec:sosuli}
    & \Square\\
  21 & Resultater fra registeret er tilpasset og tilgjengelig for pasienter
    & \ref{sec:respas} & \Square\\
  22 & Viser at registeret har ført til kvalitetsforbedring/endret klinisk
  praksis & \ref{sec:evakva} & \Square\\
  \label{tab:sta} 	 
\end{longtable}




\end{document}
