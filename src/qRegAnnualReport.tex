\documentclass[norsk, a4paper, twocolumn]{report}

\usepackage[utf8x]{inputenc}
\usepackage[colorlinks=true]{hyperref}
\usepackage{babel}
\usepackage{authblk}
\usepackage{lipsum}
\usepackage{longtable}
\usepackage{multicol}
\usepackage{color}

% Sett registerets navn
\def \registernavn {\textit{Navn på register}}

%% tittelting
\title{\registernavn \\ \textbf{Aarsrapport \textit{[årstall]}\\
Plan for forbedringstiltak \textit{[årstall+1]}}}

\author[1]{Ola Normann}
\author[2]{Medel Svensson}
\author[1]{Kari Normann}
\affil[1]{et Sykehus, et Foretak, et Sted}
\affil[2]{en annen adresse, et annet Sted}

\renewcommand\Authands{ og }
\renewcommand\Authfont{\scshape}
\renewcommand\Affilfont{\itshape\small}

% stil for veiledende tekst
\definecolor{guidegray}{rgb}{0.2,0.2,0.2}
\newcommand{\guide}[1] {
	\textit{[\textcolor{guidegray}{#1}]}
	}
%% ---


\begin{document}


\maketitle


\onecolumn
%\frontmatter



\chapter*{Om årsrapportmalen}
Etablering av en mal for årsrapport for anvendelse av de nasjonale medisinske
kvalitetsregistre gjøres på bestilling av den interregionale styringsgruppen
(styringsgruppen).
SKDE står for det praktiske arbeidet med malen, og innholdet er basert på
en rekke vedtak gjort i styringsgruppen samt innspill til justering fra
relevante aktører i registermiljøet.

Ved bruk av malen til etablering av faktiske årsrapporter skal informasjonen
som inngår i dette kapittelet fjernes.

Malen vil være et levende dokument som
forvaltes av SKDE som fortløpende innarbeider alle vedtatte endringer. Siste
versjon av malen vil således kunne fås ved henvendelse til SKDE. Dokumentet
kan fritt distribueres.

\begin{table}[ht]
  \centering
  \begin{tabular}{lrp{8cm}l}
    \hline
    Versjon & Dato & Logg & Ansvarlig \\
    \hline
    0.1 & 24. januar 2013 & Opprettet første gang & Are Edvardsen \\
    0.2 & 1. februar 2013 & Endret etter innspill fra nodemøtet &
    Are Edvardsen \\
    0.3 & 14. mars 2013 & Endret etter innspill fra Leif Ivar Havelin,
    Svein Rotevatn,
    Reinhard Seifert, Sandra Julsen Hollung, Gro Andersen, Tore Solberg og
    Anne Marie Fenstad & Are Edvardsen \\
    0.4 & 2. april 2013 & Mindre endringer etter første møte i ekspertgruppen &
    Are Edvardsen \\
    0.5 & 7. mai 2013 & Etter interne innspill, restrukturering til 1)
    årsrapport og 2) planlagte tiltak. Mer veiledende tekst. Del ang.
    foretaksspesifikk rapportering er tatt ut av dokumentet & Are Edvardsen \\
    0.6 & 16. mai 2013 & Endring i beskrivelse av
    dekningsgrad, noen nye referanser og sammenslåing i generell
    registerbeskrivelse & Are Edvardsen \\
    \hline
  \end{tabular}
  \caption{Endringslogg for dette dokumentet. Gjeldende versjon er siste
  oppføring i denne tabellen.}
  \label{tab:log}
\end{table}


%\mainmatter

\tableofcontents
\part{Årsrapport}\label{par:rap}

\chapter{Oppsummering}


%\onecolumn

% for å se hvordan dette kan ta seg ut i tabellform, lag en fylltekst
\def \fyll {På den ene siden går det stadig oppover, men på den andre går
det lukt åt skogen. Ellers er det mest på det jevne og litt rundtomkring\ldots}

\begin{longtable}{lp{5cm}p{8cm}}
  \caption[Sammendrag av status for \registernavn]
  {Sammendrag av status for \registernavn. Detaljert beskrivelse finnes i hvert
  enkelt av kapitlene som tabellen viser til.} \\
    \hline
    Kap. & Kategori & Sammendrag av status \\
    \hline
    \endfirsthead
    \caption[]{forts.}\\
    \hline
    Kap.  & Kategori & Sammendrag av status \\
    \hline
    \endhead
    \hline
    \endfoot
    \hline
    \endlastfoot
    & \multicolumn{2}{l}{\textbf{\nameref{cha:reg}}} \\
    \ref{sec:bak} & \nameref{sec:bak} & \fyll \\
    \ref{sec:for} & \nameref{sec:for} & \fyll\\
    \ref{cha:led} & \nameref{cha:led} & \fyll\\
    \ref{cha:jur} & \nameref{cha:jur} & \fyll\\[8pt]
     & \multicolumn{2}{l}{\textbf{\nameref{cha:dek}}} \\
    \ref{sec:reg} & \nameref{sec:reg} & \fyll \\
%    \ref{sec:enh} & \nameref{sec:enh} & \fyll \\
    \ref{sec:met} & \nameref{sec:met} & \fyll \\
    \ref{sec:endek} & \nameref{sec:endek} & \fyll \\
    \ref{sec:obs} & \nameref{sec:obs} & \fyll \\
%    \ref{sec:endek} & \nameref{sec:endek} & \fyll \\
%    \ref{sec:ind} & \nameref{sec:ind} & \fyll \\[8pt]
     & \multicolumn{2}{l}{\textbf{\nameref{cha:kva}}} \\
    \ref{sec:sik} & \nameref{sec:sik} & \fyll \\
    \ref{sec:metval} & \nameref{sec:metval} & \fyll \\
    \ref{sec:valdat} & \nameref{sec:valdat} & \fyll \\[8pt]
     & \multicolumn{2}{l}{\textbf{\nameref{cha:dat}}} \\
    \ref{sec:metfan} & \nameref{sec:metfan} & \fyll \\
    \ref{sec:resfag} & \nameref{sec:resfag} & \fyll \\
    \ref{sec:resled} & \nameref{sec:resled} & \fyll \\
    \ref{sec:respas} & \nameref{sec:respas} & \fyll \\
    \ref{sec:off} & \nameref{sec:off} & \fyll \\[8pt]
     & \multicolumn{2}{l}{\textbf{\nameref{cha:fag}}} \\ 
    \ref{sec:regspe} & \nameref{sec:regspe} & \fyll \\
%    \ref{sec:nasret} & \nameref{sec:nasret} & \fyll \\
    \ref{sec:retut} & \nameref{sec:retut} & \fyll \\
    \ref{sec:retbru} & \nameref{sec:retbru} & \fyll \\
    \ref{sec:brures} & \nameref{sec:brures} & \fyll \\
    \ref{sec:evakva} & \nameref{sec:evakva} & \fyll \\
    \ref{sec:kom} & \nameref{sec:kom} & \fyll \\[8pt]
     & \multicolumn{2}{l}{\textbf{\nameref{cha:for}}} \\     
    \ref{sec:samfag} & \nameref{sec:samfag} & \fyll \\
    \ref{sec:vitarb} & \nameref{sec:vitarb} & \fyll \\[8pt]
     & \multicolumn{2}{l}{\textbf{\nameref{cha:pas}}} \\
    \ref{sec:pasutk} & \nameref{sec:pasutk} & \fyll \\
    \ref{sec:demuli} & \nameref{sec:demuli} & \fyll \\
    \ref{sec:sosuli} & \nameref{sec:sosuli} & \fyll \\

  \label{tab:sta}
\end{longtable}

\twocolumn


%\part{Registerbeskrivelse}\label{par:reg}


\chapter{Registerbeskrivelse}\label{cha:reg}
\guide{Alt i dette kapitlet hentes fra protokoll, søknad om nasjonal status,
konsesjonssøknad etc.}


\section{Rasjonale}\label{cha:ras}
\subsection{Bakgrunn for registeret}\label{sec:bak}
\subsection{Registerets formål}\label{sec:for}

\section{Faglig ledelse og databehandleransvar}\label{cha:led}
\section{Juridisk hjemmelsgrunnlag}\label{cha:jur}

%\chapter{Rasjonale}\label{cha:ras}

%\section{Bakgrunn for registeret}\label{sec:bak}
%\guide{Hentes fra protokoll, søknad om nasjonal status, konsesjonssøknad etc.}

%\section{Registerets formål}\label{sec:for}
%\guide{Hentes fra protokoll, søknad om nasjonal status, konsesjonssøknad etc.}

%\chapter{Faglig ledelse og databehandleransvar}\label{cha:led}
%\guide{Hentes fra protokoll, søknad om nasjonal status, konsesjonssøknad etc.}

%\chapter{Juridisk hjemmelsgrunnlag}\label{cha:jur}
%\guide{Hentes fra protokoll, søknad om nasjonal status, konsesjonssøknad etc.}



%\part{Status og evaluering}


\chapter{Dekningsgrad}\label{cha:dek}
%\guide{Status og evaluering av registerets dekningsgrad gis under og et
%sammendrag av dette fins i Tabell \ref{tab:sta}. Med dekningsgrad menes et
%forhold mellom faktiske observasjoner i registeret for en gitt egenskap og det
%man basert på uavhengige kilder vet eller antar skal være et totalt antall
%mulige observasjoner. Eksempelvis kan et register motta registreringer fra
%tre avdelinger mens det totale antall avdelinger som kunne ha registrert
%i Norge er fire. Dekningsgrad på avdelingsnivå vil derfor være på
%75\%. I tillegg brukes her begrepet HENDELSE som eksempelvis for
%registre innen kirurgi kan vise til INNGREP eller som for registre
%for kroniske tilstander kan vise til KONTAKT eller FORLØP.}

%Forslag fra Alex:
\guide{En dekningsgrad oppgis som regel på institusjons- og observasjonsnivå på 
årlig basis. For eksempel kan et register ett år motta registreringer fra tre
av totalt fire mulige institusjoner/avdelinger slik at dekningsgraden på
institusjonsnivå er 75\%, mens dekningsgraden på observasjonsnivå 
(som kanskje er mer interessant) angir
forholdet mellom registrerte observasjoner og det man basert på uavhengige
kilder (f.eks.
\href{http://www.kvalitetsregistre.no/dekningsgradsanalyser/category358.html}
{NPR}\footnote{\url{http://www.kvalitetsregistre.no/dekningsgradsanalyser/category358.html}})
vet eller antar skal være det totale antall mulige
observasjoner i det aktuelle året. Observasjonene kan avhengig av registerets
natur eksempelvis være kirurgiske prosedyrer, diagnoser, individer med en 
kronisk tilstand eller hver enkelt hendelse hos individer med en kronisk
tilstand.}

\section{Antall registreringer}\label{sec:reg}
\guide{Per 31. desember for rapporteringsåret, angi totalt antall
observasjoner per avdeling (inkludert avdelinger uten observasjoner). Kan
også brytes opp i antall per år.}

\section{Metode for beregning av dekningsgrad}\label{sec:met}
\guide{Angi kilder/registre som er brukt, for hvilken periode beregningene
gjelder for og hvordan analysene er gjort (metode).}

\section{Dekningsgrad på avdelingsnivå}\label{sec:endek}
\guide{Angi resultatet av dekningsgradsanalysen på enhets/avdelingsnivå,
eventuelt også endringer fra tidligere år.}

%\section{Dekningsgrad på hendelsesnivå}\label{sec:ind}
%\guide{Angi resultat av dekningsgradsanalyse på hendelsesnivå,
%eventuelt (antatte) endringer fra tidligere år. Hva som skal karakteriseres
%som en hendelse vil variere mellom ulike typer registre. Se veiledningen i
%begynnelsen av dette kapitlet.}

\section{Dekningsgrad på observasjonsnivå}\label{sec:obs}
\guide{Angi resultat av dekningsgradsanalyse på observasjonsnivå. Hva som
karakteriserer en observasjon vil variere fra register til register. For et
register for kroniske tilstander vil dekningsgrad kunne beregnes for
antall registrerte personer samt for enkeltobservasjoner for hver person.}

\chapter{Datakvalitet}\label{cha:kva}
\guide{Status og evaluering av registerets  datakvalitet. Sammendrag
gis i Tabell \ref{tab:sta}.}

\section{Metoder for intern sikring av datakvalitet}\label{sec:sik}
\guide{Beskriv infrastruktur, rutiner og/eller punktaksjoner, eksempelvis
opplæring, informasjonsarbeid, tekniske støttesystemer og tilbakekobling mellom
register og innregistrerende enheter.}

\section{Metode for validering av data i registeret}\label{sec:metval}
\guide{Beskriv hvilke eksterne kilder (registre) som er benyttet og hvordan
dette er sammenholdt med registerets data (design/subsampling).} 

\section{Vurdering av datakvalitet}\label{sec:valdat}
\guide{Angi viktige funn og en vurdering av resultatene fra
valideringsstudien slik som funn av systematiske feil, tilfeldige feil og
antatt effekt dette har på validitet for hele registeret.}



\chapter{Datainnsamling og formidling av resultater}\label{cha:dat}
\guide{Status og evaluering av registerets  datafangst og formidling av 
resultater. Et sammendrag av dette gis i Tabell \ref{tab:sta}.}

\section{Metoder for fangst av data}\label{sec:metfan}
\guide{Beskriv hva/hvem som registrerer, eksempelvis sykehusenes egne
fagsystemer, helsearbeidere, pasienter og hvordan innsamlingen skjer,
eksempelvis elektronisk eller på papirskjema.}

\section{Resultater til deltagende fagmedisinske enheter}\label{sec:resfag}
\guide{Beskriv form og frekvens,  eventuelt tilbakemeldinger på relevans,
endringsønsker mm.}

\section{Resultater til administrasjon og ledelse}\label{sec:resled}
\guide{Beskriv form og frekvens og gi en vurdering av om, og hvordan denne
informasjonen brukes av ledelse og administrasjon.}

\section{Resultater til pasienter}\label{sec:respas}
\guide{Beskriv form og frekvens og eventuelt tilbakemeldinger fra pasienter,
særlig i forhold til nytten av informasjonen fra registeret.}

\section{Offentliggjøring av resultater på sykehusnivå}\label{sec:off}
\guide{Beskriv hvor/til hvem, form og frekvens.}




\chapter{Fagutvikling og kvalitetsforbedring av tjenesten}\label{cha:fag}
\guide{Beskrivelse av registerets fagutvikling og kvalitetsforbedringstiltak
i rapporteringsperioden. Sammendrag av dette fins i Tabell \ref{tab:sta}.}

\section{Registerets spesifikke kvalitetsmål}\label{sec:regspe}
\guide{Beskriv de mål/variabler/indikatorer som angir grad av kvalitet.
Splittes opp i strukturelle-, prosessuelle- og resultatmessige mål. Se også
\href{http://www.kvalitetsregistre.no/getfile.php/Norsk/Dokumenter/Handbok_120411_siste.pdf}{Registerhåndboka}
\footnote{\url{http://www.kvalitetsregistre.no/getfile.php/Norsk/Dokumenter/Handbok_120411_siste.pdf}}
for ytterligere beskrivelse.}


\subsection{Strukturmål}
\guide{Strukturdimensjonen beskriver ressurser og organisering (for eksempel 
økonomi, personalressurser, fysiske rammer, medisinsk utstyr,
forekomst av kliniske retningslinjer) en har til disposisjon.
Eksempler på strukturvariabler:
\begin{itemize}
  \item Hvor mange sykepleiere med spesialkompetanse er ansatt ved avdelingen?
  \item Hvor mange senger er tilgjengelig ved avdelingen?
  \item Hvilket mottakerapparat finnes i kommunen? 
\end{itemize}
}

\subsection{Prosessmål}
\guide{Prosessdimensjonen beskriver de aktiviteter helsetjenesten utfører
overfor pasienten. Eksempler på prosessvariabler:
\begin{itemize}
  \item Får pasienter med gitt diagnose behandling i tråd med anerkjente
    retningslinjer?
  \item Hvilken diagnostisk utredning for en gitt tilstand gjøres ved sykehuset?
  \item Hvilken behandling får pasientene med en gitt diagnose?
  \item Hvor lang ventetid er det fra vurdering av henvisning til undersøkelse?
  \item Hvor mange døgn er pasientene innlagt for en gitt tilstand?
\end{itemize}
}


\subsection{Resultatmål}
\guide{De viktigste målene på kvalitet i helsetjenesten
vil være hvordan det faktisk går med pasientene.
Resultatmål må være mulig å tolke på en
relativt enkel måte. Noen eksempler på resultatvariabler:
\begin{itemize}
  \item Pasientrapporterte utkommemål, som
    opplevd smerte og livskvalitet for en gitt
    tilstand
    før og etter operasjon
  \item Symptomscore før og etter gitt behandling
  \item Opplevd nytte av behandling
  \item 30-dagers dødelighet
  \item Bivirkninger
  \item Komplikasjoner og uønskede hendelser
  \item Arbeidsførhet etter en gitt behandling
  \item Funksjonsnivå i dagliglivets gjøremål etter endt rehabilitering
\end{itemize}
I den grad registeret har etablert
(nasjonale) kvalitetsindikatorer er dette eksempel på resultatmål.}

% ref: http://www.kvalitetsregistre.no/getfile.php/Norsk/Dokumenter/Handbok_120411_siste.pdf

\section{Bidrag til utvikling av nasjonale retningslinjer}\label{sec:retut}
\guide{I den grad det er relevant, beskriv hvordan registeret bidrar til
etablering av nasjonale retningslinjer}

\section{Bruk av nasjonale retningslinjer}\label{sec:retbru}
\guide{I den grad de eksisterer, beskriv i hvor stor grad deltagende enheter
etterlever etablerte nasjonale retningslinjer}

%\section{Evaluering av nasjonale retningslinjer -- tas ut}\label{sec:nasret}
%\guide{Beskriv de nasjonale retningslinjer som evalueres av registeret???}

%\section{I hvor stor grad tas retningslinjene i bruk?}\label{sec:retbru}
%\guide{Beskriv i hvor stor grad de nasjonale retningslinjene er tatt i
%aktiv bruk.}

\section{Deltagende enheters bruk av resultater til egen kvalitetsforbedring}\label{sec:brures}
\guide{I enheter som registrerer inn til registeret, angi konkrete prosjekter
for kvalitetsforbedring basert på resultater fra registeret. Beskriv
hvordan registeret ved styringsgruppen bidrar til oppstart og gjennomføring
av kvalitetsforbedringsarbeid lokalt hos innregistrerende enheter.}

\section{Evaluering av kvalitetsforbedrende tiltak (endret praksis)}\label{sec:evakva}
\guide{Beskriv resultater av gjennomførte tiltak for kvalitetsforbedringer.}

\section{Komplikasjoner og uheldige hendelser}\label{sec:kom}
\guide{Beskriv hva som registreres av komplikasjoner og/eller uheldige
hendelser i forbindelse med den behandling som registeret omfatter.}

\chapter{Forskning og internasjonalt samarbeid}\label{cha:for}
\guide{Status og evaluering av registerets vitenskapelige aktivitet og
internasjonalt samarbeid. Et sammendrag av dette gis i Tabell 
\ref{tab:sta}.}

\section{Samarbeid med søster-registre eller andre fagmiljø utenfor Norge}\label{sec:samfag}
\guide{Beskriv eventuelle samarbeid registeret har med tilsvarende miljøer
utenfor Norge.}

\section{Vitenskapelige arbeider}\label{sec:vitarb}
\guide{Angi bidrag fra registeret i fagfellevurderte tidsskrifter,
vitenskapelige konferanser, utdanning av doktorgradsstipendiater eller
lignende.}


\chapter{Pasientrapportering, demografi og sosiale forhold}\label{cha:pas}
\guide{Status og evaluering av registerets bruk av pasientrapporterte
resultater samt
hvordan resultater fra registeret varierer med demografiske og sosiale forhold
i pasientpopulasjonen. Et sammendrag av dette gis i Tabell \ref{tab:sta}.}

\section{Pasientrapporterte resultatmål}\label{sec:pasutk}
\guide{Beskriv hvilke instrumenter som benyttes og gi et sammendrag av de
viktigste funn. Vis til del \ref{part:res} av årsrapporten.}

\section{Demografiske ulikheter}\label{sec:demuli}
\guide{Angi hvilke variabler som er tilgjengelig i registeret (alder, kjønn,
bosted) og viktige funn i forhold til registerets kvalitetsmål. Vis til del
\ref{part:res} av årsrapporten.}

\section{Sosiale ulikheter}\label{sec:sosuli}
\guide{Beskriv hvilke målinger som benyttes (sivilstatus, utdanning, inntekt,
etnisitet) og viktige funn i forhold til registerets kvalitetsmål. Vis til del
\ref{part:res} i årsrapporten.}


%\part{Registerfaglige resultater (tradisjonell årsrapport)}\label{part:res}
\chapter{Registerfaglige resultater (tradisjonell årsrapport)}\label{part:res}


\guide{Fyll inn resultater. Relativt uavhengig ift resten av dokumentet. Dette
er ``Årsrapporten'' i tradisjonell forstand.}

%\chapter{Første tema}
%\lipsum

%\chapter{Andre tema}
%\lipsum

%\chapter{Tredje tema}
%\lipsum






%\part{Tiltak}

\onecolumn
\part{Plan for forbedringstiltak}
\chapter{Momentliste}
\guide{Basert på beskrevet status og evaluering av registeret, beskriv de
tiltak som er planlagt gjennomført fra og med neste rapporteringsperiode.
Punktlisten under kan brukes som et utgangspunkt og hjelp til hva som kan
inngå i en slik beskrivelse. For ytterligere forklaring av de ulike punktene,
se respektive kapitler under Del \ref{par:rap}: \nameref{par:rap}.}

\begin{itemize}
  \item Dekningsgrad
    \begin{itemize}
      \item Nye registrerende enheter/avdelinger
      \item Forbedring av dekningsgrad for andre observasjoner i registeret
    \end{itemize}
  \item Datakvalitet
    \begin{itemize}
      \item Endringer av rutiner for intern kvalitetssikring av data
      \item Oppfølging av resultater fra validering mot eksterne kilder
    \end{itemize}
  \item Datainnsamling og formidling av resultater
    \begin{itemize}
      \item Forbedring av metoder for fangst av data
      \item Forbedring av resultatformidling til fagmedisinske enheter
      \item Forbedring av resultatformidling til administrasjon og ledelse
      \item Forbedring av resultatformidling til pasienter
      \item Forbedring av hvordan resultater på sykehusnivå offentliggjøres
    \end{itemize}
  \item Fagutvikling og kvalitetsforbedring av tjenesten
    \begin{itemize}
      \item Nye kvalitetsmål
      \item Bidrag til etablering av nasjonale retningslinjer
      \item Registrerende enheters etterlevelse av nasjonale retningslinjer
      \item Økt bruk av resultater til kvalitetsforbedring i hver enkelt enhet
      \item Prioriterte, faglige forbedringsområder
    \end{itemize}
  \item Forskning og internasjonalt samarbeid
    \begin{itemize}
      \item Nye, internasjonale samarbeidspartnere
      \item Forskningsprosjekter og annen vitenskapelig aktivitet
    \end{itemize}
  \item Pasientrapportering, demografi og sosiale forhold
    \begin{itemize}
      \item Nye pasientrapporterte resultater som skal inn i registeret
      \item Utvidet bruk av pasientrapporterte resultater
      \item Nye demografiske variabler som skal inn i registeret
      \item Utvidet bruk av demografiske variabler
      \item Nye variabler som beskriver sosiale forhold hos pasientene
      \item Utvidet bruk av resultater som beskriver sosiale ulikheter
    \end{itemize}
\end{itemize}

% \chapter{Dekningsgrad}
% \fyll
% 
% \section{Nye registrerende enheter}
% \fyll
% 
% \section{Økning av dekningsgrad på hendelsesnivå}
% \fyll
% 
% \chapter{Datakvalitet}
% \fyll
% 
% \section{Endringer av intern kvalitetssikring av data}
% \fyll
% 
% \section{Oppølging av resultater fra validering mot eksterne kilder}
% \fyll
% 
% \chapter{Datainnsamling og formidling av resultater}
% \fyll
% 
% \section{Forbedring av metoder for fangst av data}
% \fyll
% 
% \section{Forbedring av resultatformidling til fagmedisinske enheter}
% \fyll
% 
% \section{Forbedring av resultatformidling til administrasjon og ledelse}
% \fyll
% 
% \section{Forbedring av resultatformidling til pasienter}
% \fyll
% 
% \section{Forbedring av hvordan resultater på sykehusnivå offentliggjøres}
% \fyll
% 
% \chapter{Fagutvikling og kvalitetsforbedring av tjenesten}
% \fyll
% 
% \section{Nye kvalitetsmål}
% \fyll
% 
% \section{Nye nasjonale retningslinjer}
% \fyll
% 
% \section{Implementering av retningslinjer}
% \fyll
% 
% \section{Økt bruk av resultater til kvalitetsforbedring i hver enkelt enhet}
% \fyll
% 
% \section{Prioriterte, faglige forbedringsområder}
% \fyll
% 
% \chapter{Forskning og internasjonalt samarbeid}
% \fyll
% 
% \section{Nye, internasjonale samarbeidspartnere}
% \fyll
% 
% \section{Forskningsprosjekter og vitenskapelig aktivitet}
% \fyll
% 
% \chapter{Pasienterfaring, demografi og sosiale forhold}
% \fyll
% 
% \section{Nye pasientrapporterte resultater som skal inn i registeret}
% \fyll
% 
% \section{Utvidet bruk av pasientrapporterte resultater}
% \fyll
% 
% \section{Nye demografiske variabler som skal inn i registeret}
% \fyll
% 
% \section{Utvidet bruk av demografiske variabler}
% \fyll
% 
% \section{Nye variabler som beskriver sosiale forhold hos pasienten}
% \fyll
% 
% \section{Utvidet bruk av resultater som beskriver sosiale ulikheter}
% \fyll
% 




%\listoffigures
\listoftables


\end{document}
