%%% Årsrapportmal for nasjonale medisinske kvalitetsregistre.               %%%
%                                                                             %
%   For utfyllende informasjon om dokumentet, se 'Om årsrapportmalen'.        %
%   SKDE vedlikeholder LaTeX-koden som er under versjonskontroll i            %
%   kodebrønn (Subversion) hos Helse Nord IKT.                                %
%                                                                             %
%%% SKDE, Are Edvardsen 2013, 2014, 2015, 2016                              %%%


% Suggested scheme for converitng to MS Word file:

% # compile latex source
% latex qRegAnnualReport.tex

% # make html from latex source
% htlatex qRegAnnualReport.tex

% # Next steps in libreoffice
% -open qRegAnnualReport.html
% -export as openoffice (odt)
% -open new odt-file
% -remove comments
% -set page size to A4
% -fix superscript on front page
% -remove footnote references that points to local html-file
% -adjust table
% -add page breaks
% -save as doc/docx
% -check that all is ok


% For print, bruk dokumentklassen 'book'. Bruk 'report' for å unngå blanke
% sider
\documentclass[norsk, a4paper, twocolumn]{report}

\usepackage[utf8x]{inputenc}
\usepackage{babel}
\usepackage{authblk}
\usepackage{longtable}
\usepackage{multicol}
\usepackage{color}
\usepackage{wasysym}
\usepackage[raggedright]{titlesec}
\usepackage{color,hyperref}
\usepackage{soul}

% Gjør alle lenker mørkeblå
\definecolor{darkblue}{rgb}{0.0,0.0,0.3}
\hypersetup{colorlinks,breaklinks,linkcolor=darkblue,urlcolor=darkblue,anchorcolor=darkblue,citecolor=darkblue}

% Ikke sett inn sidenummer for ``Parts''
\makeatletter
\let\sv@endpart\@endpart
\def\@endpart{\thispagestyle{empty}\sv@endpart}
\makeatother

% Ikke bruk innrykk for nye avsnitt, men heller litt større vertikal avstand
\setlength{\parindent}{0pt}
\setlength{\parskip}{1ex plus 0.5ex minus 0.2ex}


% Sett registerets navn
\def \registernavn {\textit{Navn på register}}


% tittelting
\title{\registernavn \\ \textbf{Årsrapport for \textit{[årstall]} med \\
plan for forbedringstiltak}}

\author[1]{Ola Normann}
\author[2]{Medel Svensson}
\author[1]{Kari Normann}
\affil[1]{et Sykehus, et Foretak, et Sted}
\affil[2]{en annen adresse, et annet Sted}

\renewcommand\Authands{ og }
\renewcommand\Authfont{\scshape}
\renewcommand\Affilfont{\itshape\small}


% stil for veiledende tekst
\definecolor{guidegray}{rgb}{0.2,0.2,0.2}
\newcommand{\guide}[1] {
	\textit{[\textcolor{guidegray}{#1}]}
	}


% overstyre mulige ord-deling-er
\hyphenation{stadium-vurder-ing system-atiske pasi-ent-rap-porterte
retnings-linjer hjemmelsgrunnlag data-behandlings-ansvar nasjon-ale
indi-kator-er styrings-gruppe}


\begin{document}

\maketitle

\onecolumn


% Ta ut hele kapittelet ved bruk til faktisk rapport.
% Av en eller annen grunn ønskes også versjonsoversikten tatt ut fra malen\ldots
% Tas inn igjen ved å fjerne løkka rundt

\iffalse
\chapter*{Om årsrapportmalen}
Etablering av en mal for årsrapport for anvendelse av de nasjonale medisinske
kvalitetsregistre gjøres på bestilling av den interregionale styringsgruppen
(styringsgruppen).
SKDE står for det praktiske arbeidet med malen, og innholdet er basert på
en rekke vedtak gjort i styringsgruppen samt innspill til justering fra
relevante aktører i registermiljøet.

Ved bruk av malen til etablering av faktiske årsrapporter skal informasjonen
som inngår i dette kapittelet fjernes.

Malen vil være et levende dokument som
forvaltes av SKDE som fortløpende innarbeider alle vedtatte endringer. Siste
versjon av malen vil således kunne fås ved henvendelse til SKDE. Dokumentet
kan fritt distribueres. For full utnyttelse av malen bør årsrapporten
produseres i \LaTeX. Til dette kan man eksempelvis bruke programvaren
\href{http://texstudio.sourceforge.net/}{TexStudio}\footnote{http://texstudio.sourceforge.net/}
som er fritt tilgjengelig og gratis i bruk.

\begin{longtable}{lrp{8cm}p{2.5cm}}
  \caption{Endringslogg for dette dokumentet. Gjeldende versjon er siste
  oppføring i denne tabellen.} \\
  \hline
    Versjon & Dato & Aktivitet & Ansvarlig \\
    \hline
    \endfirsthead
    \caption[]{forts.} \\
    \hline
    Versjon & Dato & Aktivitet & Ansvarlig \\
    \hline
    \endhead
    \\
    \multicolumn{4}{c}{\textit{Tabellen fortsetter på neste side}} \\
    \hline
    \endfoot
    \hline
    \endlastfoot
    0.1 & 24. januar 2013 & Opprettet første gang & Are Edvardsen \\
    0.2 & 1. februar 2013 & Endret etter innspill fra nodemøtet &
    Are Edvardsen \\
    0.3 & 14. mars 2013 & Endret etter innspill fra Leif Ivar Havelin,
    Svein Rotevatn,
    Reinhard Seifert, Sandra Julsen Hollung, Gro Andersen, Tore Solberg og
    Anne Marie Fenstad & Are Edvardsen \\
    0.4 & 2. april 2013 & Mindre endringer etter første møte i ekspertgruppen &
    Are Edvardsen \\
    0.5 & 7. mai 2013 & Etter interne innspill, restrukturering til 1)
    årsrapport og 2) planlagte tiltak. Mer veiledende tekst. Del ang.
    foretaksspesifikk rapportering er tatt ut av dokumentet & Are Edvardsen \\
    0.6 & 16. mai 2013 & Endring i beskrivelse av
    dekningsgrad, noen nye referanser og sammenslåing i generell
    registerbeskrivelse & Are Edvardsen \\
    0.7 & 4. juni 2013 & En hel del endinger og forenklinger etter diskusjon i
    ekspertgruppa 31. mai. Blant annet er sammendrag i tabellform med lenker
    til resten av dokumentet tatt ut & Eva Stensland, Are Edvardsen \\
    0.9 & 13. juni 2013 & Minimale endringer etter inspill fra møte i
    interregional styringsgruppe 12. juni & Eva Stensland \\
    0.91 & 18. juni 2013 & Nytt kapittel med generell bakgrunn og veiledning.
    Mindre endringer på form & Eva Stensland, Are Edvarsen \\
    1.0 & 18. juni 2013 & For bruk i årsrapporter for 2012 & Are Edvardsen \\
     & & & \\
    1.1 & 3. april 2014 & Endringer ift revidert stadieinndeling. Reversering
    av tidligere endringer: flytte Resultater frem og ta tilbake
    tabularisk sammendrag for hjelp til vurdering av stadium & Are Edvardsen,
    Eva Stensland \\
    1.2 & 4. april 2014 & Tabularisk sammendrag erstattet med stadiumvurdering
    i egen Del med referanser til øvrige deler av dokumentet & Are Edvardsen,
    Eva Stensland \\
    1.3 & 7. april 2014 & Gjennomgang og justering hvert pkt under
    stadieinndelingen. Små endringer i dokumentet forøvrig &
    Eva Stensland, Philip Skau, Gøril Nordgård, Are Edvardsen \\
    1.4 & 24. april 2014 & Små tekstlige og kosmetiske endringer &
    Philip Skau, Are Edvardsen \\
    1.5 & 9. mai 2014 & Større endringer etter behandling i ekspertgruppa
    29. og 30. april 2014 & Eva Stensland, Are Edvardsen \\
    1.6 & 15. mai 2014 & Mindre språklige endringer og oppdatering av
    momentliste ihht ny struktur & Eva Stensland, Are Edvardsen \\
    1.7 & 6. juni 2014 & Lagt inn eksempel på CheckedBox for stadier og info
    i dokumentkoden, inkludert konvertering til msword & Are Edvardsen \\
    2.0 & 11. juni 2014 & Endringer etter innspill og vedtak i styringsgruppen
    4. juni 2014. For bruk i årsrapporter for 2013 & Eva Stensland,
    Are Edvardsen \\
    2.1 & 3. oktober 2014 & Rettelse av strukturell feil etter innspill fra
    Reinhard Seifert & Are Edvardsen \\
    2.2 & 7. mai 2015 & Lagt til Summary samt en del justering og tillegg
    til forklaringstekster. Lagt til nytt punkt om "Inklusjonskriterier" i
    kapittel \ref{cha:fag}. & Are Edvardsen \\
    2.3 & 15. april 2016 & Foreløpige endringer. & Eva Stensland mfl. \\
    2.4 & 28. juni 2016 & Oppdaterte lenker og tekst. Lagt til avkryssing
    for "Ikke aktuelt" i tabell for stadievurdering & Eva Stensland \\
    3.0 & 29. juni 2016 & Klar for bruk i årsrapporter for 2015.
    Versjonshåndtering vil bli flyttet fra subversion til git med 
    tilgjengeliggjøring gjennom GitHub. & Are Edvardsen \\
    3.1 & 13. juni 2017 & Endringer bedt om av interregional styringsgruppe,
    ekspertgruppen og servicemiljøet i Helse Vest. Eksperimentell del om
    dataaktualitet (datakvalitet) er også lagt til & Are Edvardsen \\
    3.2 & 20. juni 2017 & Eksperimentell del om dataaktualitet er tatt ut fordi
    den (foreløpig) ikke følges opp gjennom stadievurderingen & Are Edvardsen \\
  \label{tab:log}
\end{longtable}
\fi




\chapter*{Bakgrunn og veiledning til utfylling}


\section*{Bakgrunn}
En årsrapport fra et medisinsk kvalitetsregister bør utarbeides først og
fremst for å vise hvilken nytte helsetjenesten har hatt av resultatene fra
registeret, og
hvordan registeret kan brukes til klinisk kvalitetsforbedringsarbeid.
Årsrapporten bør utformes slik
at den også kan leses og forstås av personer utenfor det aktuelle fagmiljø.

Malen for årsrapport er utarbeidet av Nasjonalt servicemiljø for
kvalitetsregistre på bestilling av interregional styringsgruppe, for bruk av alle
nasjonale medisinske kvalitetsregistre. Malen inneholder sentrale
rapporteringselementer som blant annet har sitt utgangspunkt i
\href{https://www.kvalitetsregistre.no/artikkel/stadieinndeling}{stadieinndelingssystemet}\footnote{\url{https://www.kvalitetsregistre.no/artikkel/stadieinndeling}}
for kvalitetsregistre, og en resultatdel.

Mottaker for årsrapporten er det enkelte registers RHF. For å kunne gi en
samlet oversikt over nasjonale kvalitetsregistres
årsrapporter, samt å være grunnlag for publisering av resultater fra
kvalitetsregistrene, ber vi om at kopi av rapporten også sendes SKDE innen
innleveringsfristen.
\href{https://www.kvalitetsregistre.no/ekspertgruppen}{Ekspertgruppen}\footnote{\url{https://www.kvalitetsregistre.no/ekspertgruppen}}
vil gjøre en gjennomgang av alle årsrapportene for inneværende
årsrapportperiode, og kategorisere de nasjonale kvalitetsregistrene i henhold
til stadieinndelingssystemet.


\section*{Veiledning til utfylling}
Kapittel \ref{cha:metoder}-\ref{cha:for} i malen er beskrivende, og utfylles så
langt det er mulig. Det vil
være mange registre som mangler informasjon for utfylling av ett eller flere
underkapitler. Ved manglende informasjon lar man det aktuelle underkapitlet
stå tomt. Det er laget en veiledende tekst til alle underkapitler som har
som hensikt å beskrive hvilken informasjon man ønsker fylt inn. I kapittel
\ref{cha:kva} og \ref{cha:dat}
er begrepet ”institusjon” brukt. Her fyller registeret inn informasjon på
foretaks-, sykehus- eller avdelingsnivå avhengig av hvilken informasjon som
er tilgjengelig i hvert enkelt register. 

Kapittel \ref{cha:res} er resultatdelen av årsrapporten, og her fyller det
enkelte register inn de resultater (tabeller, figurer og tekst) de ønsker å
formidle. Det er et krav at man viser resultater fra de viktigste
kvalitetsindikatorer i registeret, og at resultatene formidles på
sykehusnivå.

I hver helseregion finnes det representanter for det nasjonale
servicemiljøet for medisinske kvalitetsregistre som kan svare på spørsmål
angående årsrapporter. Kontaktinformasjon til disse finnes på
servicemiljøets \href{https://www.kvalitetsregistre.no/om-nasjonalt-servicemiljo-medisinske-kvalitetsregistre}{nettsider}\footnote{\url{https://www.kvalitetsregistre.no/om-nasjonalt-servicemiljo-medisinske-kvalitetsregistre}}.




\tableofcontents




\part{Årsrapport}\label{par:rap}
\thispagestyle{empty}
\twocolumn



\chapter{Sammendrag/Summary}
\guide{Kortfattet sammendrag av de viktigste elementer man fra registerets side
ønsker å formidle i årsrapporten. De viktigste resultater for
behandlingskvalitet og kvalitetsforbedringstiltak bør angis.}

\section*{Summary in English}
\guide{Optionally, provide a summary of the annual report. Highlights from
quality assessment and improvements are relevant here}


\chapter{Registerbeskrivelse}\label{cha:reg}
\guide{Informasjon til dette kapitlet hentes fra egen registerbeskrivelse,
søknad om nasjonal status, konsesjonssøknad etc.}

\section{Bakgrunn og formål}
\subsection{Bakgrunn for registeret}\label{sec:bak}
\subsection{Registerets formål}\label{sec:for}

\section{Juridisk hjemmelsgrunnlag}\label{cha:jur}
\guide{Er registeret basert på konsesjon eller forskrift? I tilfelle
konsesjon, angi eventuell tidsbegrensning for denne.}

\section{Faglig ledelse og databehandlingsansvar}\label{cha:led}
\subsection{Aktivitet i\\fagråd/referansegruppe}
\guide{Antall møter, viktige saker som er behandlet mv. Beskriv sammensetning av
fagrådet og angi spesifikt om det finnes pasient- eller brukerrepresentant i
fagrådet.}




\chapter{Resultater}\label{cha:res}
\guide{Tekst, tabeller og figurer der resultater fra registeret
presenteres. Kapitlet bes strukturert slik at resultater for sentrale
kvalitetsindikatorer relevante for klinisk kvalitetsforbedring og
resultater viktige for vurdering av stadieinndeling presenteres først.}




\chapter{Metoder for fangst av data}\label{cha:metoder}
\guide{Beskriv hva/hvem som registrerer (eksempelvis sykehusenes egne
fagsystemer, helsearbeidere, pasienter) og hvordan innsamlingen skjer
(eksempelvis elektronisk eller på papirskjema).}




\chapter{Metodisk kvalitet}\label{cha:kva}
\guide{Status og evaluering av registerets  datakvalitet.
Dekningsgraden angir
forholdet mellom registrerte individer og det man basert på uavhengige
kilder (f.eks.
\href{https://www.kvalitetsregistre.no/malpopulasjon-og-dekningsgrad\#idx-1}
{NPR}\footnote{\url{https://www.kvalitetsregistre.no/malpopulasjon-og-dekningsgrad\#idx-1}})
vet eller antar skal være det totale antall mulige registrerbare
individer i det aktuelle året. Validitet angir graden av korrekthet for én eller
et sett av opplysninger gjerne basert på en sammenligning mot andre, uavhengige
kilder.}

\section{Antall registreringer}\label{sec:reg}
\guide{Per 31. desember for rapporteringsåret, angi antall
individer/hendelser totalt og per institusjon. Kan også brytes opp i
antall per år.}

\section{Metode for beregning av dekningsgrad}\label{sec:met}
\guide{Angi kilder/registre som er brukt, for hvilken periode beregningene
gjelder for og hvordan analysene er gjort (metode).}

\section{Tilslutning}\label{sec:endek}
\guide{Oppgi om registeret samler data fra HF i alle regioner og andel
av aktuelle sykehus/avdelinger som sender data til registeret.}

\section{Dekningsgrad}\label{sec:obs}
\guide{Angi resultat av dekningsgradsanalyse på individnivå for hele registeret
og tidspunkt for gjennomføring av dekningsgradsanalysen.}

\section{Prosedyrer for intern sikring av datakvalitet}\label{sec:sik}
\guide{Beskriv infrastruktur og rutiner (eksempelvis
opplæring, informasjonsarbeid, tekniske støttesystemer) som ivaretar
riktige og komplette data fra innregistrerende enheter og i registeret
sentralt.}

\section{Metode for validering av data i registeret}\label{sec:metval}
\guide{Beskriv hvilke eksterne kilder som er benyttet og hvordan
dette er sammenholdt med registerets data. Beskriv metoden for
valideringsstudien.} 

\section{Vurdering av datakvalitet}\label{sec:valdat}
\guide{Angi viktige funn og en vurdering av resultatene fra
valideringsanalyser slik som funn av systematiske feil, tilfeldige feil og
antatt effekt dette har på validitet for hele registeret. Angi graden av
kompletthet for sentrale variabler.}




\chapter{Fagutvikling og klinisk kvalitetsforbedring}\label{cha:fag}
\guide{Beskrivelse av registerets fagutvikling og kliniske
kvalitetsforbedringstiltak
i rapporteringsperioden, herunder også status og evaluering av
pasientrapporterte resultater og hvordan resultater fra registeret varierer
med demografiske og sosiale forhold i pasientpopulasjonen.}

\section{Pasientgruppe som omfattes av registeret}
\guide{Angi inklusjonskriterier for registeret, eksempelvis definert av
diagnosekoder og/eller prosedyrekoder.}

\section{Registerets spesifikke kvalitetsindikatorer}\label{sec:regspe}
\guide{Beskriv de viktigste variabler/indikatorer som angir grad av
kvalitet (struktur-, prosess- og resultatmål). Se også
\href{https://www.kvalitetsregistre.no/registerarbeid}{kvalitetsregistre.no}\footnote{\url{https://www.kvalitetsregistre.no/registerarbeid}}
for ytterligere beskrivelse. Indikér hvilke av disse som vil egne seg som
nasjonale kvalitetsindikatorer.}

\section{Pasientrapporterte resultat- og erfaringsmål (PROM og PREM)}\label{sec:pasutk}
\guide{Beskriv hvilke instrumenter som benyttes eller angi hvor langt
registeret har kommet for å inkludere slike varaibler. Resultater i seg selv
beskrives i kap. \ref{cha:res}.}

\section{Sosiale og demografiske ulikheter i helse}\label{sec:sosdem}
\guide{Angi hvilke analyser som eventuelt er utført. Resultater i seg selv
beskrives i kap. \ref{cha:res}.}

\section{Bidrag til utvikling av nasjonale retningslinjer, nasjonale
kvalitetsindikatorer o.l.}\label{sec:retut}
\guide{I den grad det er relevant, beskriv hvordan registeret bidrar til
etablering av nasjonale retningslinjer, handlingsplaner/-program}

\section{Etterlevelse av nasjonale retningslinjer}\label{sec:retbru}
\guide{Dersom nasjonale retningslinjer finnes, beskriv om disse er inkludert
som variabler i registeret. Der nasjonale retningslinjer ikke finnes bør
\normalfont{best practice}
\textit{beskrives. Resultater i seg selv beskrives i kap. \ref{cha:res}.}}

\section{Identifisering av kliniske forbedringsområder}\label{sec:ide}
\guide{Beskriv kliniske forbedringsområder som er identifisert på bakgrunn
av analyser fra registeret.}

\section{Tiltak for klinisk kvalitetsforbedring initiert av
registeret}\label{sec:brures}
\guide{Angi konkrete prosjekter
for klinisk kvalitetsforbedring initiert av registeret. Beskriv
hvordan registeret bidrar til oppstart og gjennomføring
av klinisk kvalitetsforbedringsarbeid lokalt hos innregistrerende
institusjoner.}

\section{Evaluering av tiltak for klinisk kvalitetsforbedring (endret praksis)}\label{sec:evakva}
\guide{Beskriv resultater av gjennomførte tiltak for kvalitetsforbedringer
i kap. \ref{sec:brures}.}

\section{Pasientsikkerhet}\label{sec:kom}
\guide{Beskriv hva som registreres av komplikasjoner og/eller uønskede
hendelser i forbindelse med den behandling som registeret omfatter. I den grad
registeret rapporterer til pasientsikkerhetsprogrammet, tas dette med.}




\chapter{Formidling av resultater}\label{cha:dat}
\guide{Status og evaluering av formidlingsform og -frekvens
av resultater fra registeret. Resultater i seg selv beskrives i kap.
\ref{cha:res}.}

\section{Resultater tilbake til deltakende fagmiljø}\label{sec:resfag}
\guide{Beskriv form og frekvens,  eventuelt tilbakemeldinger på relevans,
endringsønsker mm.}

\section{Resultater til administrasjon og ledelse}\label{sec:resled}
\guide{Beskriv form og frekvens, og hvordan denne
informasjonen formidles til ledelse og administrasjon.}

\section{Resultater til pasienter}\label{sec:respas}
\guide{Beskriv form og frekvens av tilpasset informasjon og eventuelt
tilbakemeldinger fra pasienter, særlig i forhold til nytten av informasjonen
fra registeret.}

\section{Publisering av resultater på institusjonsnivå}\label{sec:off}
\guide{Beskriv hvor/til hvem, form og frekvens. Resultater i seg selv beskrives i kap. \ref{cha:res}.}




\chapter{Samarbeid og forskning}\label{cha:for}
\guide{Status og evaluering av samarbeid med andre registre/fagmiljø og  vitenskapelig aktivitet.}

\section{Samarbeid med andre helse- og kvalitetsregistre}\label{sec:samfag}
\guide{Beskriv eventuelle samarbeid registeret har med andre registre eller relevante fagmiljø, nasjonalt eller internasjonalt.}

\section{Vitenskapelige arbeider}\label{sec:vitarb}
\guide{Angi bidrag fra registeret i fagfellevurderte tidsskrifter,
vitenskapelige konferanser, utdanning av doktorgradsstipendiater eller
lignende i løpet av de siste to år.}




\onecolumn


\part{Plan for forbedringstiltak}\label{par:for}


\chapter{Forbedringstiltak}
\guide{Basert på beskrevet status og evaluering av registeret, beskriv de
tiltak som er gjennomført inneværende år samt tiltak som er planlagt gjennomført  for neste kalenderår.
Punktlisten under kan brukes som et utgangspunkt og hjelp til hva som kan
inngå i en slik beskrivelse. For ytterligere forklaring av de ulike punktene,
se respektive kapitler under Del \ref{par:rap}: \nameref{par:rap}.}

\begin{itemize}
  \item Datafangst
    \begin{itemize}
      \item Forbedring av metoder for fangst av data
    \end{itemize}
  \item Metodisk kvalitet
    \begin{itemize}
      \item Nye registrerende enheter/avdelinger
      \item Forbedring av dekningsgrad i registeret
      \item Endringer av rutiner for intern kvalitetssikring av data
      \item Oppfølging av resultater fra validering mot eksterne kilder
    \end{itemize}
  \item Fagutvikling og kvalitetsforbedring av tjenesten
    \begin{itemize}
      \item Nye kvalitetsindikatorer
      \item Nye pasientrapporterte resultater som skal inn i registeret
      \item Utvidet bruk av pasientrapporterte resultater
      \item Nye demografiske variabler som skal inn i registeret
      \item Utvidet bruk av demografiske variabler
      \item Bidrag til etablering av nasjonale retningslinjer eller 
        nasjonale kvalitetsindikatorer
      \item Registrerende enheters etterlevelse av nasjonale retningslinjer
      \item Økt bruk av resultater til klinisk kvalitetsforbedring i hver
        enkelt institusjon
      \item Prioriterte, kliniske forbedringsområder
    \end{itemize}
  \item Formidling av resultater
    \begin{itemize}
      \item Forbedring av resultatformidling til deltagende fagmiljø
      \item Forbedring av resultatformidling til administrasjon og ledelse
      \item Forbedring av resultatformidling til pasienter
      \item Forbedring av hvordan resultater på institusjonsnivå
        publiseres
    \end{itemize}
  \item Samarbeid og forskning
    \begin{itemize}
      \item Nye samarbeidspartnere
      \item Forskningsprosjekter og annen vitenskapelig aktivitet
    \end{itemize}
\end{itemize}




\part{Stadievurdering}


\chapter{Referanser til vurdering av stadium}
\guide{Oversikt over vurderingspunkter som legges til grunn for
\href{https://www.kvalitetsregistre.no/node/704}{stadieinndeling av registre}\footnote{\url{https://www.kvalitetsregistre.no/node/704}}
med referanser til relevant informasjon
gitt i årsrapporten. Denne delen fylles ut og er ment som en
hjelp til registeret og ekspertgruppen i vurdering av registeret. Stadium 1
er oppfylt når registeret har status som nasjonalt.}

\bigskip
\bigskip

\begin{longtable}{rp{8cm}lccc}
  \caption[Vurderingspunkter for stadium \registernavn]
  {Vurderingspunkter for stadium \registernavn} \\
  \hline 	 
  Nr & Beskrivelse & Kapittel & Ja & Nei & Ikke aktuell\\ 	 
  \hline 	 
  \endfirsthead 	 
  \caption[]{forts.}\\ 	 
  \hline 	 
  Nr & Beskrivelse & Kapittel & Ja & Nei & Ikke aktuell\\
  \hline 	 
  \endhead
  \\
  \multicolumn{4}{c}{\textit{Tabellen fortsetter på neste side}} \\
  \hline
  \endfoot 	 
  \hline 	 
  \endlastfoot
   & \textbf{Stadium 2} & & \\
  1 & Er i drift og samler data fra HF i alle helseregioner
    & \ref{cha:res}, \ref{sec:endek}  & \CheckedBox
    & \Square & \Square \\
  2 & Presenterer resultater på nasjonalt nivå & \ref{cha:res} & \Square
    & \Square & \Square \\
  3 & Har en konkret plan for gjennomføring av dekningsgradsanalyser
    & \ref{sec:met} & \Square & \Square & \Square \\
  4 & Har en konkret plan for gjennomføring av analyser og løpende
      rapportering av resultater på sykehusnivå tilbake til deltakende
      enheter & \ref{sec:resfag}, \ref{sec:resled} & \Square& \Square
      & \Square \\
  5 & Har en oppdatert plan for videre utvikling av registeret
    & Del \ref{par:for} & \Square& \Square & \Square \\
    & & & & \\

   & \textbf{Stadium 3} & & \\
  6 & Kan redegjøre for registerets datakvalitet
    & \ref{sec:sik} & \Square& \Square & \Square \\
  7 & Har beregnet dekningsgrad mot uavhengig datakilde
    & \ref{sec:met}, \ref{sec:endek}, \ref{sec:obs} & \Square& \Square
      & \Square \\
  8 & Har dekningsgrad over 60 \%
    & \ref{sec:obs} & \Square& \Square & \Square \\
  9 & Registrerende enheter kan få
      utlevert egne aggregerte og nasjonale resultater
    & \ref{sec:resfag}, \ref{sec:resled}  & \Square & \Square & \Square \\
  10 & Presenterer deltakende enheters etterlevelse av de viktigste
      nasjonale retningslinjer der disse finnes
    & \ref{sec:retbru} & \Square & \Square & \Square \\
  11 & Har identifisert kliniske forbedringsområder basert på analyser fra
       registeret & \ref{sec:ide} & \Square & \Square & \Square \\
  12 & Brukes til klinisk kvalitetsforbedringsarbeid
    & \ref{sec:brures}, \ref{sec:evakva} & \Square & \Square & \Square \\
  13 & Resultater anvendes vitenskapelig & \ref{sec:vitarb} & \Square
    & \Square & \Square \\
  14 & Presenterer resultater for PROM/PREM & \ref{sec:pasutk} & \Square
    & \Square & \Square \\
  15 & Har en oppdatert plan for videre utvikling av registeret
    & Del \ref{par:for} & \Square & \Square & \Square \\
    & & & & \\

   & \textbf{Stadium 4} & & \\
  16 & Kan dokumentere registerets datakvalitet gjennom valideringsanalyser
    & \ref{sec:metval}, \ref{sec:valdat} & \Square & \Square & \Square \\
  17 & Presenterer oppdatert dekningsgradsanalyse hvert 2. år
    & \ref{sec:met}, \ref{sec:endek}, \ref{sec:obs} & \Square & \Square
      & \Square \\
  18 & Har dekningsgrad over 80\% & \ref{sec:obs} & \Square & \Square
      & \Square \\
  19 & Registrerende enheter har løpende (on-line) tilgang til oppdaterte
      egne og nasjonale resultater & \ref{sec:resfag}
      & \Square & \Square & \Square \\
  20 & Resultater fra registeret er tilpasset og tilgjengelig for pasienter
    & \ref{sec:respas} & \Square & \Square & \Square \\
  21 & Kunne dokumentere at registeret har ført til
      kvalitetsforbedring/endret klinisk praksis & \ref{sec:evakva}
      & \Square& \Square & \Square \\
  \label{tab:sta} 	 
\end{longtable}


%\listoffigures
%\listoftables


\end{document}
