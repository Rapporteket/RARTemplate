%%% Årsrapportmal for nasjonale medisinske kvalitetsregistre.               %%%
%                                                                             %
%   For utfyllende informasjon om dokumentet, se 'Om årsrapportmalen'.        %
%   SKDE vedlikeholder LaTeX-koden som er under versjonskontroll på           %
%   github (https://github.com/Rapporteket/RARTemplate).                      %
%                                                                             %
%%% SKDE, Are Edvardsen 2013-2020                                           %%%


% Suggested scheme for converitng to MS Word file:

% # compile latex source
% latex qRegAnnualReport.tex

% # make html from latex source
% htlatex qRegAnnualReport.tex

% # Next steps in libreoffice
% -open qRegAnnualReport.html
% -export as openoffice (odt)
% -open new odt-file
% -remove comments
% -set page size to A4
% -fix superscript on front page
% -remove footnote references that points to local html-file
% -adjust table
% -add page breaks
% -save as doc/docx
% -check that all is ok


% For print, bruk dokumentklassen 'book'. Bruk 'report' for å unngå blanke
% sider
\documentclass[norsk, a4paper, twocolumn]{report}

\usepackage[utf8x]{inputenc}
\usepackage{babel}
\usepackage{authblk}
\usepackage{longtable}
\usepackage{microtype}
\usepackage{multicol}
\usepackage{color}
\usepackage{wasysym}
\usepackage[raggedright]{titlesec}
\usepackage[breaklinks]{hyperref}
\usepackage[color]{changebar}

% fix feil i v 2.10.1 av titlesec?
\usepackage{etoolbox}
\makeatletter
\patchcmd{\ttlh@hang}{\parindent\z@}{\parindent\z@\leavevmode}{}{}
\patchcmd{\ttlh@hang}{\noindent}{}{}{}
\makeatother

% Gjør alle lenker mørkeblå
\definecolor{darkblue}{rgb}{0.0,0.0,0.3}
\hypersetup{colorlinks,breaklinks,linkcolor=darkblue,urlcolor=darkblue,anchorcolor=darkblue,citecolor=darkblue}

% Ikke sett inn sidenummer for ``Parts''
\makeatletter
\let\sv@endpart\@endpart
\def\@endpart{\thispagestyle{empty}\sv@endpart}
\makeatother

% Ikke bruk innrykk for nye avsnitt, men heller litt større vertikal avstand
\setlength{\parindent}{0pt}
\setlength{\parskip}{1ex plus 0.5ex minus 0.2ex}


% Sett registerets navn
\def \registernavn {\textit{Navn på register}}


% tittelting
\title{\registernavn \\ \textbf{Årsrapport for \textit{[årstall]} med \\
plan for forbedringstiltak}}

\author[1]{Ola Normann}
\author[2]{Medel Svensson}
\author[1]{Kari Normann}
\affil[1]{et Sykehus, et Foretak, et Sted}
\affil[2]{en annen adresse, et annet Sted}

\renewcommand\Authands{ og }
\renewcommand\Authfont{\scshape}
\renewcommand\Affilfont{\itshape\small}


% stil for inngående tekst
\newcommand{\newtext}[1]{\cbstart\textcolor{blue}{#1\cbend}}

% stil for utgående tekst
\newcommand{\oldtext}[1]{\cbdelete\textcolor{red}{#1}}

% stil for veiledende tekst
\definecolor{guidegray}{rgb}{0.2,0.2,0.2}
\newcommand{\guide}[1]{
  \itshape[\color{guidegray}{\raggedright Veiledning -- SLETTES VED UTFYLLING: #1}]
  \normalfont
}


% overstyre mulige ord-deling-er
\hyphenation{stadium-vurder-ing system-atiske pasi-ent-rap-porterte
retnings-linjer hjemmelsgrunnlag data-behandlings-ansvar nasjon-ale
indi-kator-er styrings-gruppe validerings-studien elek-tro-nisk}


\begin{document}

\maketitle

\onecolumn


% Ta ut hele kapittelet ved bruk til faktisk rapport.
% Av en eller annen grunn ønskes også versjonsoversikten tatt ut fra malen...
% Tas inn igjen ved å fjerne løkka rundt

\iffalse
\chapter*{Om årsrapportmalen}
Etablering av en mal for årsrapport for anvendelse av de nasjonale medisinske
kvalitetsregistre gjøres på bestilling av den interregionale styringsgruppen
(styringsgruppen).
SKDE står for det praktiske arbeidet med malen, og innholdet er basert på
en rekke vedtak gjort i styringsgruppen samt innspill til justering fra
relevante aktører i registermiljøet.

Ved bruk av malen til etablering av faktiske årsrapporter skal informasjonen
som inngår i dette kapittelet fjernes.

Malen vil være et levende dokument som
forvaltes av SKDE som fortløpende innarbeider alle vedtatte endringer. Siste
versjon av malen vil således kunne fås ved henvendelse til SKDE. Dokumentet
kan fritt distribueres. For full utnyttelse av malen bør årsrapporten
produseres i \LaTeX. Til dette kan man eksempelvis bruke programvaren
\href{http://texstudio.sourceforge.net/}{TexStudio}\footnote{http://texstudio.sourceforge.net/}
som er fritt tilgjengelig og gratis i bruk.

\begin{longtable}{lrp{8cm}p{2.5cm}}
  \caption{Endringslogg for dette dokumentet. Gjeldende versjon er siste
  oppføring i denne tabellen.} \\
  \hline
    Versjon & Dato & Aktivitet & Ansvarlig \\
    \hline
    \endfirsthead
    \caption[]{forts.} \\
    \hline
    Versjon & Dato & Aktivitet & Ansvarlig \\
    \hline
    \endhead
    \\
    \multicolumn{4}{c}{\textit{Tabellen fortsetter på neste side}} \\
    \hline
    \endfoot
    \hline
    \endlastfoot
    0.1 & 24. januar 2013 & Opprettet første gang & Are Edvardsen \\
    0.2 & 1. februar 2013 & Endret etter innspill fra nodemøtet &
    Are Edvardsen \\
    0.3 & 14. mars 2013 & Endret etter innspill fra Leif Ivar Havelin,
    Svein Rotevatn,
    Reinhard Seifert, Sandra Julsen Hollung, Gro Andersen, Tore Solberg og
    Anne Marie Fenstad & Are Edvardsen \\
    0.4 & 2. april 2013 & Mindre endringer etter første møte i ekspertgruppen &
    Are Edvardsen \\
    0.5 & 7. mai 2013 & Etter interne innspill, restrukturering til 1)
    årsrapport og 2) planlagte tiltak. Mer veiledende tekst. Del ang.
    foretaksspesifikk rapportering er tatt ut av dokumentet & Are Edvardsen \\
    0.6 & 16. mai 2013 & Endring i beskrivelse av
    dekningsgrad, noen nye referanser og sammenslåing i generell
    registerbeskrivelse & Are Edvardsen \\
    0.7 & 4. juni 2013 & En hel del endinger og forenklinger etter diskusjon i
    ekspertgruppa 31. mai. Blant annet er sammendrag i tabellform med lenker
    til resten av dokumentet tatt ut & Eva Stensland, Are Edvardsen \\
    0.9 & 13. juni 2013 & Minimale endringer etter inspill fra møte i
    interregional styringsgruppe 12. juni & Eva Stensland \\
    0.10 & 18. juni 2013 & Nytt kapittel med generell bakgrunn og veiledning.
    Mindre endringer på form & Eva Stensland, Are Edvarsen \\
    1.0 & 18. juni 2013 & For bruk i årsrapporter for 2012 & Are Edvardsen \\
     & & & \\
    1.1 & 3. april 2014 & Endringer ift revidert stadieinndeling. Reversering
    av tidligere endringer: flytte Resultater frem og ta tilbake
    tabularisk sammendrag for hjelp til vurdering av stadium & Are Edvardsen,
    Eva Stensland \\
    1.2 & 4. april 2014 & Tabularisk sammendrag erstattet med stadiumvurdering
    i egen Del med referanser til øvrige deler av dokumentet & Are Edvardsen,
    Eva Stensland \\
    1.3 & 7. april 2014 & Gjennomgang og justering hvert pkt under
    stadieinndelingen. Små endringer i dokumentet forøvrig &
    Eva Stensland, Philip Skau, Gøril Nordgård, Are Edvardsen \\
    1.4 & 24. april 2014 & Små tekstlige og kosmetiske endringer &
    Philip Skau, Are Edvardsen \\
    1.5 & 9. mai 2014 & Større endringer etter behandling i ekspertgruppa
    29. og 30. april 2014 & Eva Stensland, Are Edvardsen \\
    1.6 & 15. mai 2014 & Mindre språklige endringer og oppdatering av
    momentliste ihht ny struktur & Eva Stensland, Are Edvardsen \\
    1.7 & 6. juni 2014 & Lagt inn eksempel på CheckedBox for stadier og info
    i dokumentkoden, inkludert konvertering til msword & Are Edvardsen \\
    2.0 & 11. juni 2014 & Endringer etter innspill og vedtak i styringsgruppen
    4. juni 2014. For bruk i årsrapporter for 2013 & Eva Stensland,
    Are Edvardsen \\
    2.1 & 3. oktober 2014 & Rettelse av strukturell feil etter innspill fra
    Reinhard Seifert & Are Edvardsen \\
    2.2 & 7. mai 2015 & Lagt til Summary samt en del justering og tillegg
    til forklaringstekster. Lagt til nytt punkt om "Inklusjonskriterier" i
    kapittel \ref{cha:fag}. & Are Edvardsen \\
    2.3 & 15. april 2016 & Foreløpige endringer. & Eva Stensland mfl. \\
    2.4 & 28. juni 2016 & Oppdaterte lenker og tekst. Lagt til avkryssing
    for "Ikke aktuelt" i tabell for stadievurdering & Eva Stensland \\
    3.0 & 29. juni 2016 & Klar for bruk i årsrapporter for 2015.
    Versjonshåndtering vil bli flyttet fra subversion til git med 
    tilgjengeliggjøring gjennom GitHub. & Are Edvardsen \\
    3.1 & 13. juni 2017 & Endringer bedt om av interregional styringsgruppe,
    ekspertgruppen og servicemiljøet i Helse Vest. Eksperimentell del om
    dataaktualitet (datakvalitet) er også lagt til & Are Edvardsen \\
    3.2 & 20. juni 2017 & Eksperimentell del om dataaktualitet er tatt ut fordi
    den (foreløpig) ikke følges opp gjennom stadievurderingen & Are Edvardsen \\
    4.0 & 20. juni 2017 & Klar for bruk i årsrapporter for 2016 & Are Edvardsen
    \\
    4.1 & 4. mai 2018 & Endringer i veiledende tekst til årsrapporter for 2017
    vist med utgående tekst i rødt og ny tekst i blått & Are Edvardsen, Lena
    Ringstad Olsen \\
    5.0 & 4. mai 2018 & For bruk i årsrapporter for 2017 med utvidet
    veiledningstekst, særlig i kapitlene 'Datakvalitet' og 'Fagutvikling og
    klinisk kvalitetsforbedring' & Are Edvardsen \\
    5.1 & 14. mai 2018 & Oppmykning av krav, bla fjerning av krav 20 i stadium
    4 & Eva Stensland \\
    6.0 & 26. mars 2019 & For bruk i årsrapporter for 2018 med endinger i
    veiledningstekst og noen nye underkapitler & Are Edvardsen \\
    6.1 & 27. mars 2019 & Etter korrektur & Are Edvardsen \\
    7.0 & 7. februar 2020 & Årlig oppdatering & Marianne Nicolaisen \\
    7.1 & 11. februar 2020 & Feilrettinger & Marianne Nicolaisen \\
    7.2 & 11. februar 2020 & Tilgjengelig for redigering i Overleaf &
    Are Edvardsen \\
    7.3 & 13. juli 2020 & Egenmelding om oppfølging fra forrige år & Marianne
    Nicolaisen

  \label{tab:log}
\end{longtable}
\fi




\chapter*{Bakgrunn og veiledning til utfylling}


\section*{Bakgrunn}
En årsrapport fra et medisinsk kvalitetsregister bør utarbeides først og
fremst for å vise hvilken nytte helsetjenesten har hatt av resultatene fra
registeret, og hvordan registeret kan brukes til pasientrettet
kvalitetsforbedringsarbeid. Årsrapporten bør utformes slik at den også kan leses
og forstås av personer utenfor det aktuelle fagmiljø.

Malen for årsrapport er utarbeidet av Nasjonalt servicemiljø for
kvalitetsregistre på bestilling av interregional styringsgruppe, for bruk av
alle nasjonale medisinske kvalitetsregistre. Malen inneholder sentrale
rapporteringselementer som blant annet har sitt utgangspunkt i
\href{https://www.kvalitetsregistre.no/artikkel/stadieinndeling}{stadieinndelingssystemet}\footnote{\url{https://www.kvalitetsregistre.no/artikkel/stadieinndeling}}
for kvalitetsregistre.

Mottaker for årsrapporten er det enkelte registers RHF. For å kunne gi en
samlet oversikt over nasjonale kvalitetsregistres årsrapporter, samt å være
grunnlag for publisering av resultater fra kvalitetsregistrene, ber vi om at
kopi av rapporten også sendes SKDE innen innleveringsfristen.
\href{https://www.kvalitetsregistre.no/ekspertgruppen}{Ekspertgruppen}\footnote{\url{https://www.kvalitetsregistre.no/ekspertgruppen}}
vil gjøre en gjennomgang av alle årsrapportene for inneværende
årsrapportperiode, og kategorisere de nasjonale kvalitetsregistrene i henhold
til stadieinndelingssystemet.


\section*{Veiledning til utfylling}
Datagrunnlaget for årsrapporten er data innsamlet i rapporteringsåret.

Kapittel \ref{cha:res} er resultatdelen av årsrapporten, og her fyller det
enkelte register inn de resultater (tabeller, figurer og tekst) de ønsker å
formidle. Det er et krav at man viser resultater fra de viktigste
kvalitetsindikatorer i registeret, og at resultatene formidles på enhetsnivå.

Kapittel \ref{cha:metoder}-\ref{cha:for} i malen er beskrivende, og utfylles så
langt det er mulig. Det vil være mange registre som mangler informasjon for
utfylling av ett eller flere underkapitler. Ved manglende informasjon lar man
det aktuelle underkapitlet stå tomt. Det er laget en veiledende tekst til alle
underkapitler som har som hensikt å beskrive hvilken informasjon man ønsker fylt
inn. I kapittel \ref{cha:kva} og \ref{cha:dat} er begrepet ``enhet'' brukt.
Her fyller registeret inn informasjon på foretaks-, sykehus- eller avdelingsnivå
avhengig av hvilken informasjon som er tilgjengelig i hvert enkelt register. 

I hver helseregion finnes det representanter for det nasjonale
servicemiljøet for medisinske kvalitetsregistre som kan svare på spørsmål
angående årsrapporter. Kontaktinformasjon til disse finnes på servicemiljøets
\href{https://www.kvalitetsregistre.no/om-nasjonalt-servicemiljo-medisinske-kvalitetsregistre}{nettsider}\footnote{\url{https://www.kvalitetsregistre.no/om-nasjonalt-servicemiljo-medisinske-kvalitetsregistre}}.




\tableofcontents




\part{Årsrapport}\label{par:rap}
\thispagestyle{empty}
\twocolumn



\chapter{Sammendrag/Summary}
\guide{Kortfattet sammendrag av de viktigste elementer man fra registerets side
ønsker å formidle i årsrapporten. De viktigste resultater for
behandlingskvalitet og kvalitetsforbedringstiltak bør angis.}

\section*{Summary in English}
\guide{Optionally, provide a summary of the annual report. Highlights from
quality assessment and improvements are relevant here}


\chapter{Registerbeskrivelse}\label{cha:reg}
\guide{Informasjon til dette kapitlet hentes fra egen registerbeskrivelse,
søknad om nasjonal status, etc.}

\section{Bakgrunn og formål}
\subsection{Bakgrunn for registeret}\label{sec:bak}
\subsection{Registerets formål}\label{sec:for}
\subsection{Analyser som belyser registerets formål}\label{sec:anafor}
\guide{Beskriv analyser som er relevante for å oppfylle registerets formål.
Registeret må beskrive hvilke analyser som er tenkt gjennomført for å oppfylle
registerets formål, med hovedfokus på kvalitetsindikatorer og PROM/PREM.}

\section{Juridisk hjemmelsgrunnlag}\label{cha:jur}

\section{Faglig ledelse og dataansvar}\label{cha:led}
\guide{Angi navn på faglig leder og dataansvarlig institusjon.}
\subsection{Aktivitet i fagråd/referansegruppe}
\guide{Oppgi
\begin{itemize}
  \item antall møter,
  \item viktige saker som er behandlet mv.,
  \item medlemmer av fagrådet,
  \item hvem som er fagrådets leder,
  \item hvilke institusjoner medlemmene tilhører,
  \item om det finnes pasient- eller brukerrepresentant i fagrådet.
\end{itemize}    
}




\chapter{Resultater}\label{cha:res}
\guide{Tekst, tabeller og figurer der resultater fra registeret presenteres,
fortrinnsvis på sykehusnivå/enhetsnivå der dette er mulig. Kapitlet bes 
strukturert slik at resultater for sentrale kvalitetsindikatorer relevante
for pasientrettet kvalitetsforbedring og resultater viktige for vurdering av
stadieinndeling presenteres først.}

\section{Kvalitetsindikatorer og PROM/PREM}\label{sec:indpp}
\guide{Resultater for alle kvalitetsindikatorer, samt resultater for PROM/PREM.
\begin{itemize}
  \item I alle figurer vises resultater på enhetsnivå
  \item Vis prosenter/gjennomsnitt for hver enhet
  \item Samlet nasjonalt resultat bør inngå i figuren
\end{itemize}
}

\section{Andre analyser}\label{sec:andana}
\guide{Resultater for øvrige analyser som ikke inngår i kap. \ref{sec:indpp}.
Kan med fordel deles opp i flere underkapitler.
\begin{itemize}
  \item I alle figurer vises resultater på enhetsnivå
  \item Vis prosenter/gjennomsnitt for hver enhet
  \item Samlet nasjonalt resultat bør inngå i figuren
\end{itemize}
}

\chapter{Metoder for fangst av data}\label{cha:metoder}
\guide{Beskriv:
\begin{itemize}
  \item hva/hvem som registrerer (eksempelvis sykehusenes egne fagsystemer,
    helsearbeidere, pasienter),
  \item hvordan innsamlingen skjer (eksempelvis elektronisk eller på
    papirskjema),
  \item angi elektronisk(e) løsning(er) for datafangst.
\end{itemize}
}




\chapter{Datakvalitet}\label{cha:kva}
\guide{Status og evaluering av registerets  datakvalitet. I dette kapitlet skal
det rapporteres på følgende dimensjoner ved datakvalitet:
\begin{itemize}
  \item kompletthet (tilslutning, dekningsgrad og variabelkompletthet),
  \item korrekthet (validitet),
  \item reliabilitet (variasjon ved gjentatte målinger)
\end{itemize}
Hvis registeret har utført ytterligere dokumentasjon/undersøkelse av
datakvalitet, beskrives også dette her. For nærmere beskrivelser av ulike
datakvalitetsdimensjoner, se
\href{https://www.kvalitetsregistre.no/datakvalitet-pa-1-2-3}{Datakvalitet på 1-2-3}\footnote{\url{https://www.kvalitetsregistre.no/datakvalitet-pa-1-2-3}}.}


\section{Antall registreringer}\label{sec:reg}
\guide{Per 31. desember for rapporteringsåret, angi antall individer/hendelser
per enhet og år, samt totalt.}

\section{Metode for beregning av dekningsgrad}\label{sec:met}
\guide{Angi:
\begin{itemize}
  \item ekstern kilde/register som er brukt for beregning av dekningsgrad på
    individnivå
  \item hvilken periode beregningene gjelder for
  \item hvordan analysene er gjort (metode)
  \item dersom det ikke er hensiktsmessig å beregne dekningsgrad mot uavhengig
    kilde, skal registeret redegjøre for dette og beskrive hvilken annen metode
    som benyttes for beregning av dekningsgrad.
\end{itemize}
}

\section{Tilslutning}\label{sec:endek}
\guide{Oppgi:
\begin{itemize}
  \item om registeret samler data fra alle helseregioner
  \item andel av aktuelle sykehus/avdelinger som registrer i registeret
  \item dersom registerets pasientgruppe ikke behandles i alle regioner,
    presiseres dette
  \item hvilke enheter som skulle levert data, men som ikke gjør det.
\end{itemize}
}

\section{Dekningsgrad}\label{sec:obs}
\guide{Angi:
\begin{itemize}
  \item resultat av dekningsgradsanalyse på individnivå for alle pasientgrupper
    samlet
  \item tidspunkt for gjennomføring av dekningsgradsanalysen
  \item dekningsgrad også på sykehus- eller enhetsnivå.
\end{itemize}
}

\section{Prosedyrer for intern sikring av datakvalitet}\label{sec:sik}
\guide{Beskriv infrastruktur og rutiner (eksempelvis opplæring av registrarer,
informasjonsarbeid, logiske kontroller i innregistreringsløsning, automatiske
datakontroller, regelmessige manuelle kontroller) som ivaretar riktige og
komplette data fra innregistrerende enheter og i registeret sentralt.}

\section{Metoder for vurdering av datakvalitet}\label{sec:metval}
\guide{Angi metoder som er benyttet for beregning/vurdering av datakvalitet for
resultatene som presenteres i kap. \ref{sec:valdat}.}

\section{Vurdering av datakvalitet}\label{sec:valdat}
\guide{Angi viktige funn og en vurdering av resultatene fra undersøkelser av
variabelkompletthet, korrekthet (validitet) og reliabilitet (reproduserbarhet)
for kvalitetsindikatorene, og evt. i tillegg noen av de mest sentrale
variablene:
\begin{enumerate}
  \item kompletthet: angi graden av kompletthet for sentrale variabler og
    variabler som inngår i kvalitetsindikatorer for rapporteringsåret
  \item korrekthet: angi grad av samsvar mellom registeret og gullstandard
    (f.eks. journal) i prosent, og når undersøkelsen ble utført
  \item reliabilitet, dvs. variasjon ved gjentatte målinger: angi hvilke og hvor
    stor andel av variablene som ikke var reliable og når undersøkelsen ble
    utført
\end{enumerate}
Gi en overordnet vurdering av funnene i datakvalitetsundersøkelsene og hvilken
betydning dette har for tolkning av resultater fra registeret.}


\chapter{Fagutvikling og pasientrettet kvalitetsforbedring}\label{cha:fag}
\guide{Beskrivelse av:
\begin{itemize}
  \item registerets fagutvikling
  \item kliniske kvalitetsforbedringstiltak i rapporteringsperioden
  \item status og evaluering av pasientrapporterte resultater
  \item hvordan resultater fra registeret varierer med demografiske og sosiale
    forhold i pasientpopulasjonen.
\end{itemize}
}

\section{Pasientgruppe som omfattes av registeret}
\guide{Angi inklusjonskriterier for registeret, eksempelvis definert av
diagnosekoder og/eller prosedyrekoder.}

\section{Registerets variabler og spesifikke
kvalitetsindikatorer}\label{sec:regspe}
\guide{
\begin{itemize}
  \item Redegjør for hvordan registerets beskrivelse av registervariablene
    foreligger (f.eks kodebok)
  \item Beskriv de viktigste variabler/indikatorer som angir grad av kvalitet
    (struktur-, prosess- og resultatmål). Det skal fremgå hvordan de er
    definert, og målnivåer eller andre metoder for evaluering av måloppnåelse
    dersom målnivå ikke er satt, skal beskrives. Se evt.
\href{https://www.kvalitetsregistre.no/kvalitet-i-helsetjenesten\#kvalitetsindikatorer}{kvalitetsregistre.no}\footnote{\url{https://www.kvalitetsregistre.no/kvalitet-i-helsetjenesten\#kvalitetsindikatorer}}
for ytterligere beskrivelse.
  \item Oppgi gjerne hvilke av disse indikatorene som er, eller vil egne seg
    som, nasjonale kvalitetsindikatorer 
  \item Oppgi kliniske kvalitetsindikatorer som er relevante for å vurdere
    hvorvidt de viktigste nasjonale eller internasjonale retningslinjer (der
    disse finnes) etterleves, alternativt for å vurdere hvorvidt
    ``best practice''-anbefalinger etterleves
\end{itemize}
}

\section{Pasientrapporterte resultat- og erfaringsmål
(PROM og PREM)}\label{sec:pasutk}
\guide{
\begin{itemize}
  \item Beskriv hvilke instrumenter/skjema som brukes for rutinemessig
    innsamling av PROM/PREM, og gi begrunnelse for valg av instrument/skjema
  \item Beskriv eventuelle utfordringer knyttet til innsamling av PROM/PREM
    (eks. metodiske, tekniske, juridiske)
  \item Dersom registeret ikke har PROM/PREM - angi hvor langt registeret har
    kommet for å inkludere slike variabler
  \item Resultater presenteres/beskrives i kap. \ref{cha:res}
\end{itemize}
}

\section{Sosiale og demografiske ulikheter i helse}\label{sec:sosdem}
\guide{Angi hvilke analyser som eventuelt er utført. Resultater
presenteres/beskrives i kap. \ref{cha:res}.}

\section{Bidrag til utvikling av nasjonale retningslinjer, nasjonale
kvalitetsindikatorer o.l.}\label{sec:retut}
\guide{I den grad det er relevant, beskriv hvordan registeret bidrar til
etablering av:
\begin{itemize}
  \item nasjonale retningslinjer,
  \item handlingsplaner/-program, og/eller
  \item nasjonale kvalitetsindikatorer.
\end{itemize}
}

\section{Etterlevelse av faglige retningslinjer}\label{sec:retbru}
\guide{Beskriv (inter)nasjonale retningslinjer. Dersom (inter)nasjonale
retningslinjer finnes, beskriv om etterlevelse av
disse kan evalueres ved hjelp av variablene i registeret. Der (inter)nasjonale
retningslinjer ikke finnes bør andre faglige retningslinjer eller
``best practice'' beskrives. Resultater beskrives i kap. \ref{cha:res}.}

\section{Identifisering av pasientrettede forbedringsområder}\label{sec:ide}
\guide{Beskriv pasientrettede forbedringsområder for registerets
pasientpopulasjon som i rapporteringsåret er identifisert på bakgrunn av
analyser fra registeret.}

\section{Tiltak for pasientrettet kvalitetsforbedring}\label{sec:brures}
\guide{Beskriv konkrete eksempler på tiltak som er igangsatt,
kontinuert/videreført eller gjennomført i rapporteringsåret for pasientrettet
kvalitetsforbedring der data fra registeret benyttes. Tiltakene kan være
initiert og igangsatt/gjennomført av innregistrerende enheter/ fagmiljø eller av
registeret.}

\section{Evaluering av tiltak for pasientrettet kvalitetsforbedring (endret
praksis)}\label{sec:evakva}
\guide{
\begin{itemize}
  \item Beskriv resultater av gjennomførte tiltak som har vært igangsatt i løpet
    av de siste tre år,for pasientrettet kvalitetsforbedring for registerets
    pasientpopulasjon som er beskrevet i kap. \ref{sec:brures} (for eksempel
    målinger før og etter tiltak)
  \item Beskriv hvordan resultater og tiltak kan spres til andre
    innrapporterende enheter
\end{itemize}
}

\section{Pasientsikkerhet}\label{sec:kom}
\guide{Beskriv hva som registreres av komplikasjoner og/eller uønskede hendelser
i forbindelse med den behandling som registeret omfatter. Resultater
presenteres og beskrives i kap. \ref{cha:res}.}




\chapter{Formidling av resultater}\label{cha:dat}
\guide{Beskriv hvordan resultater fra registeret formidles til relevante
mottakere. Dersom registeret ikke er i gang med slik formidling, skal registeret
beskrive en konkret plan for gjennomføring av analyser og jevnlig rapportering
av resultater til relevante mottakere.}

\section{Resultater tilbake til deltakende fagmiljø}\label{sec:resfag}
\guide{
\begin{itemize}
  \item Beskriv form og frekvens for rapportering av resultater, for eksempel
    om registrerende enheter har løpende tilgang til egne oppdaterte resultater
    basert på data og aggregerte nasjonale resultater
  \item Beskriv registerets resultattjeneste og hva som kreves for å få tilgang
    til de beskrevne resultatene
\end{itemize}
}

\section{Resultater til administrasjon og ledelse}\label{sec:resled}
\guide{Beskriv form og frekvens på rapportering av resultater, og hvordan denne
informasjonen formidles til ledelse og administrasjon.}

\section{Resultater til pasienter}\label{sec:respas}
\guide{Beskriv form og frekvens av tilpasset informasjon og eventuelt
tilbakemeldinger fra pasienter.}

\section{Publisering av resultater på kvalitetsregistre.no}\label{sec:off}
\guide{Beskriv hvilke kvalitetsindikatorer som presenteres på nettsiden
kvalitetsregistre.no, og oppdateringsfrekvens. Resultater beskrives i kap.
\ref{cha:res}.}




\chapter{Samarbeid og forskning}\label{cha:for}
\guide{Status og evaluering av samarbeid med andre registre/fagmiljø og
vitenskapelig aktivitet.}

\section{Samarbeid med andre helse- og kvalitetsregistre}\label{sec:samfag}
\guide{Beskriv eventuelle samarbeid registeret har med andre registre eller
relevante fagmiljø, nasjonalt eller internasjonalt.}

\section{Vitenskapelige arbeider}\label{sec:vitarb}
\guide{Angi:
\begin{itemize}
  \item antall utleveringer av data til forskningsformål i rapporteringsåret, og
  \item antall og titler på publikasjoner (publikasjonsliste) som er basert på
    data fra registeret i fagfellevurderte tidsskrifter eller vitenskapelige
    konferanser i løpet av de siste to år
\end{itemize}
}




\onecolumn


\part{Plan for forbedringstiltak}\label{par:for}


\chapter{Videre utvikling av registeret}\label{cha:vid}
\guide{
\begin{itemize}
  \item Beskriv de tiltak som er gjennomført i rapporteringsåret samt tiltak som
    er planlagt gjennomført for neste kalenderår, basert på beskrevet status og
    evaluering av registeret
  \item Beskriv konkrete og realistiske tiltak for videre utvikling til neste
    stadium innen 3 år fra nåværende stadium ble oppnådd. Planen bør omfatte
    hvert punkt som per nå ikke er oppfylt i neste stadium
\end{itemize}
    
Punktlisten under er et forslag til hva som kan inngå i en
slik beskrivelse. For ytterligere forklaring av de ulike punktene, se respektive
kapitler under Del \ref{par:rap}:
\nameref{par:rap}. Det kan gjerne tas med andre tiltak enn de nevnte
forslagene.}

\begin{itemize}
  \item Datafangst
    \begin{itemize}
      \item Forbedring av metoder for fangst av data
    \end{itemize}
  \item Datakvalitet
    \begin{itemize}
      \item Nye registrerende enheter/avdelinger
      \item Forbedring av dekningsgrad i registeret
      \item Forbedring av registerets kompletthet
      \item Forbedring av rutiner for intern kvalitetssikring av data
      \item Oppfølging av resultater fra validering mot eksterne kilder
    \end{itemize}
  \item Fagutvikling og kvalitetsforbedring av tjenesten
    \begin{itemize}
      \item Nye kvalitetsindikatorer
      \item Nye variabler for pasientrapporterte resultater
      \item Utvidet bruk av pasientrapporterte resultater
      \item Nye demografiske variabler
      \item Utvidet bruk av demografiske variabler
      \item Bidrag til etablering av nasjonale retningslinjer eller 
        nasjonale kvalitetsindikatorer
      \item Registrerende enheters etterlevelse av faglige retningslinjer
      \item Identifiserte kliniske forbedringsområder
      \item Økt bruk av resultater til pasientrettet kvalitetsforbedring i hver
        enkelt institusjon/enhet
    \end{itemize}
  \item Formidling av resultater
    \begin{itemize}
      \item Forbedring av resultatformidling til deltagende fagmiljø
      \item Forbedring av resultatformidling til administrasjon og ledelse
      \item Forbedring av resultatformidling til pasienter
      \item Forbedring av hvordan resultater på institusjonsnivå
        publiseres
    \end{itemize}
  \item Samarbeid og forskning
    \begin{itemize}
      \item Nye samarbeidspartnere
      \item Forskningsprosjekter og annen vitenskapelig aktivitet
    \end{itemize}
\end{itemize}




\part{Stadievurdering}


\chapter{Referanser til vurdering av stadium}

\section{Vurderingspunkter}
\guide{Oversikt over vurderingspunkter som legges til grunn for
\href{https://www.kvalitetsregistre.no/node/704}{stadieinndeling av registre}\footnote{\url{https://www.kvalitetsregistre.no/node/704}}
med referanser til relevant informasjon gitt i årsrapporten. Denne delen fylles
ut og er ment som en hjelp til registeret og ekspertgruppen i vurdering av
registeret. Stadium 1 er oppfylt når registeret har status som nasjonalt.}

\bigskip
\bigskip

\begin{longtable}{rp{8cm}lcc}
  \caption[Vurderingspunkter for stadium \registernavn{} og registerets egen
  evaluering.]
  {Vurderingspunkter for stadium \registernavn{} og registerets egen
  evaluering.} \\
  \hline
   & & & \multicolumn{2}{c}{Egen vurdering \textit{[årstall]}} \\
  Nr & Beskrivelse & Kapittel & Ja & Nei \\ 	 
  \hline 	 
  \endfirsthead 	 
  \caption[]{forts.}\\ 	 
  \hline
   & & & \multicolumn{2}{c}{Egen vurdering \textit{[årstall]}} \\
  Nr & Beskrivelse & Kapittel & Ja & Nei \\
  \hline 	 
  \endhead
  \\
  \multicolumn{5}{c}{\textit{Tabellen fortsetter på neste side}} \\
  \hline
  \endfoot 	 
  \hline 	 
  \endlastfoot
     & \textbf{Stadium 2} & & \\
   1 & Samler data fra alle aktuelle helseregioner
     & \ref{cha:res}, \ref{sec:endek}  & \CheckedBox & \Square \\
   2 & Presenterer kvalitetsindikatorene på nasjonalt nivå & \ref{cha:res}
     & \Square & \Square \\
   3 & Har en konkret plan for gjennomføring av dekningsgradsanalyser
     & \ref{sec:met} & \Square & \Square \\
   4 & Har en konkret plan for gjennomføring av analyser og jevnlig
       rapportering av resultater på enhetsnivå tilbake til deltakende
       enheter & \ref{sec:resfag}, \ref{sec:resled} & \Square & \Square \\
   5 & Har en oppdatert plan for videre utvikling
     & Del \ref{par:for}, \ref{cha:vid} & \Square & \Square \\
     & & & \\

     & \textbf{Stadium 3} & & \\
   6 & Kan dokumentere kompletthet av kvalitetsindikatorer
     & \ref{sec:valdat} & \Square& \Square \\
   7 & Kan dokumentere dekningsgrad på minst 60 \% i løpet av siste to år
     & \ref{sec:met}, \ref{sec:obs} & \Square& \Square \\
   8 & Registeret skal minimum årlig presentere kvalitetsindikatorresultater
       interaktivt på nettsiden kvalitetsregistre.no & \ref{sec:off} & \Square
     & \Square \\ 
   9 & Registrerende enheter kan få
       utlevert eller tilgjengeliggjort egne aggregerte og nasjonale resultater
     & \ref{sec:resfag}, \ref{sec:resled}  & \Square & \Square \\
  10 & Presenterer deltakende enheters etterlevelse av de viktigste
       faglige retningslinjer
     & \ref{cha:res}, \ref{sec:retbru} & \Square & \Square \\
  11 & Har en oppdatert plan for videre utvikling av registeret 
     & Del \ref{par:for}, \ref{cha:vid} & \Square & \Square \\
     & & & \\

     & \textbf{Stadium 4} & & \\
  12 & Har i løpet av de siste 5 år dokumentert at innsamlede data er korrekte
       og reliable & \ref{sec:metval}, \ref{sec:valdat} & \Square & \Square \\
  13 & Kan dokumnetere dekningsgrad på minst 80\% i løpet av siste to år &
       \ref{sec:met}, \ref{sec:obs} & \Square & \Square \\
  14 & Registrerende enheter har tilgang til oppdaterte egne personentydige
       resultater og aggregerte nasjonale resultater & \ref{sec:resfag}
     & \Square & \Square \\
  15 & Registerets data anvendes vitenskapelig & \ref{sec:vitarb}
     & \Square& \Square \\
  16 & Presenterer resultater for PROM/PREM (der dette er mulig)
     & \ref{sec:indpp} & \Square & \Square \\
     & & & \\

     & \textbf{Nivå A} & & \\
  17 & Registeret kan dokumentere resultater fra kvalitetsforbedrende tiltak som
       har vært igangsatt i løpet av de siste tre år. Tiltakene skal være basert
       på kunnskap fra registeret & \ref{sec:evakva} & \Square & \Square \\ 
     & & & \\

     & \textbf{Nivå B} & & \\
  18 & Registeret kan dokumentere at det i rapporteringsåret har identifisert
       forbedringsområder, og at det er igangsatt eller kontinuert/videreført
       pasientrettet kvalitetsforbedringsarbeid
     & \ref{sec:ide}, \ref{sec:brures} & \Square & \Square \\
     & & & \\

     & \textbf{Nivå C} & & \\
  19 & Oppfyller ikke krav til nivå B & & \Square & \Square \\

  \label{tab:sta} 	 
\end{longtable}


\section{Registerets oppfølging av fjorårets vurdering fra ekspertgruppen}
\guide{Beskriv hvordan registeret har fulgt opp ekspertgruppens kommentarer i vurderingstekst til forrige årsrapport.}


%\listoffigures
%\listoftables


\end{document}
