\documentclass[norsk, a4paper, twocolumn]{report}

\usepackage[utf8x]{inputenc}
\usepackage{hyperref}
\usepackage{babel}
\usepackage{authblk}
\usepackage{lipsum}
\usepackage{longtable}
\usepackage{multicol}
\usepackage{color}

% Sett registerets navn
\def \registernavn {\textit{Navn på register}}

%% tittelting
\title{\registernavn \\ \textbf{Årsrapport \the\year}}

\author[1]{Ola Normann}
\author[2]{Medel Svensson}
\author[1]{Kari Normann}
\affil[1]{et Sykehus, et Foretak, et Sted}
\affil[2]{en annen adresse, et annet Sted}

\renewcommand\Authands{ og }
\renewcommand\Authfont{\scshape}
\renewcommand\Affilfont{\itshape\small}

% stil for veiledende tekst
\definecolor{guidegray}{rgb}{0.2,0.2,0.2}
\newcommand{\guide}[1] {
	\textit{[\textcolor{guidegray}{#1}]}
	}
%% ---


\begin{document}


\maketitle

\guide{Dette må du gjøre}

\onecolumn
%\frontmatter



\chapter*{Om årsrapportmalen}
Etablering av en mal for årsrapport for anvendelse av de nasjonale medisinske
kvalitetsregistre gjøre på bestilling av den interregionale styringsgruppen
(styringsgruppen).
SKDE står for det praktiske arbeidet med malen, og innholdet er basert på
en rekke vedtak gjort i styringsgruppen samt innspill til justering fra
relevante aktører i registermiljøet.

Ved bruk av malen til etablering av faktiske årsrapporter skal informasjonen
som inngår i dette kapittelet fjernes.

Malen vil være et levende dokument som
forvaltes av SKDE som fortløpende innarbeider alle vedtatte endringer. Siste
versjon av malen vil således kunne fås ved henvendelse til SKDE. Dokumentet
kan fritt distribueres.

\begin{table}[ht]
  \centering
  \begin{tabular}{lrp{8cm}l}
    \hline
    Versjon & Dato & Logg & Ansvarlig \\
    \hline
    0.1 & 24. januar 2013 & Opprettet første gang & Are Edvardsen \\
    0.2 & 1. februar 2013 & Endret etter innspill fra nodemøtet &
    Are Edvardsen \\
    0.3 & 14. mars 2013 & Endret etter innspill fra Leif Ivar Havelin,
    Svein Rotevatn,
    Reinhard Seifert, Sandra Julsen Hollung, Gro Andersen, Tore Solberg og
    Anne Marie Fenstad & Are Edvardsen \\
    0.4 & 2. april 2013 & Mindre endringer etter første møte i ekspertgruppen &
    Are Edvardsen \\
    \hline
  \end{tabular}
  \caption{Endringslogg for dette dokumentet. Gjeldende versjon er siste
  oppføring i denne tabellen.}
  \label{tab:log}
\end{table}


%\mainmatter

\tableofcontents
\part{Rapportsammendrag}

\chapter{Oppsummering}


%\onecolumn

% for å se hvordan dette kan ta seg ut i tabellform, lag en fylltekst
\def \fyll {På den ene siden går det stadig oppover, men på den andre går
det lukt åt skogen. Ellers er det mest på det jevne og litt rundtomkring\ldots}

\begin{longtable}{lp{5cm}p{8cm}}
  \caption[Sammendrag av status for \registernavn]
  {Sammendrag av status for \registernavn. Detajert beskrivelse finnes i hvert
  enkelt av kapitlene som tabellen viser til.} \\
    \hline
    Kap. & Kategori & Sammendrag av status \\
    \hline
    \endfirsthead
    \caption[]{forts.}\\
    \hline
    Kap.  & Kategori & Sammendrag av status \\
    \hline
    \endhead
    \hline
    \endfoot
    \hline
    \endlastfoot
     & \multicolumn{2}{l}{\textbf{\nameref{par:reg}}} \\
    \ref{sec:bak} & \nameref{sec:bak} & \fyll \\
    \ref{sec:for} & \nameref{sec:for} & \fyll\\
    \ref{cha:led} & \nameref{cha:led} & \fyll\\
    \ref{cha:jur} & \nameref{cha:jur} & \fyll\\[8pt]
     & \multicolumn{2}{l}{\textbf{\nameref{cha:dek}}} \\
    \ref{sec:reg} & \nameref{sec:reg} & \fyll \\
    \ref{sec:enh} & \nameref{sec:enh} & \fyll \\
    \ref{sec:met} & \nameref{sec:met} & \fyll \\
    \ref{sec:endek} & \nameref{sec:endek} & \fyll \\
    \ref{sec:ind} & \nameref{sec:ind} & \fyll \\[8pt]
     & \multicolumn{2}{l}{\textbf{\nameref{cha:kva}}} \\
    \ref{sec:sik} & \nameref{sec:sik} & \fyll \\
    \ref{sec:metval} & \nameref{sec:metval} & \fyll \\
    \ref{sec:valdat} & \nameref{sec:valdat} & \fyll \\[8pt]
     & \multicolumn{2}{l}{\textbf{\nameref{cha:dat}}} \\
    \ref{sec:metfan} & \nameref{sec:metfan} & \fyll \\
    \ref{sec:resfag} & \nameref{sec:resfag} & \fyll \\
    \ref{sec:resled} & \nameref{sec:resled} & \fyll \\
    \ref{sec:respas} & \nameref{sec:respas} & \fyll \\ [8pt]
     & \multicolumn{2}{l}{\textbf{\nameref{cha:fag}}} \\ 
    \ref{sec:regspe} & \nameref{sec:regspe} & \fyll \\
    \ref{sec:nasret} & \nameref{sec:nasret} & \fyll \\
    \ref{sec:retbru} & \nameref{sec:retbru} & \fyll \\
    \ref{sec:brures} & \nameref{sec:brures} & \fyll \\
    \ref{sec:evakva} & \nameref{sec:evakva} & \fyll \\[8pt]
     & \multicolumn{2}{l}{\textbf{\nameref{cha:for}}} \\     
    \ref{sec:samfag} & \nameref{sec:samfag} & \fyll \\
    \ref{sec:vitarb} & \nameref{sec:vitarb} & \fyll \\[8pt]
     & \multicolumn{2}{l}{\textbf{\nameref{cha:pas}}} \\
    \ref{sec:demuli} & \nameref{sec:demuli} & \fyll \\
    \ref{sec:sosuli} & \nameref{sec:sosuli} & \fyll \\
    \ref{sec:pasutk} & \nameref{sec:pasutk} & \fyll \\
  \label{tab:sta}
\end{longtable}

\twocolumn


\part{Registerbeskrivelse}\label{par:reg}

\chapter{Rasjonale}

\section{Bakgrunn for registeret}\label{sec:bak}

\section{Registerets formål}\label{sec:for}

\chapter{Faglig ledelse og databehandleransvar}\label{cha:led}

\chapter{Juridisk hjemmelsgrunnlag}\label{cha:jur}




\part{Status og evaluering}


\chapter{Dekningsgrad}\label{cha:dek}
Status og evaluering av registerets dekningsgrad er gitt under og et sammendrag
av dette fins i Tabell \ref{tab:sta}.

\section{Antall registreringer}\label{sec:reg}
\guide{Per 31. desember for rapporteringsåret, angi totalt antall
hendelser og personer som disse fordeler seg på. Kan også brytes opp antall
per år eller endringer mellom år.}

\section{Antall registrerende enheter}\label{sec:enh}
\guide{Per 31. desember for rapporteringsåret, angi antall registrerende
enheter/avdelinger samt hvor mange som eventuelt mangler.}
\section{Metode for beregning av dekningsgrad}\label{sec:met}
\guide{Angi kilde/register som er brukt, for hvilken periode bergningen
gjelder for og hvordan analysen er gjort (metode).}
\section{Dekningsgrad på enhetsnivå}\label{sec:endek}
\guide{Kan slås sammen med kap. \ref{sec:enh}.}

\section{Dekningsgrad på hendelsesnivå}\label{sec:ind}
\guide{Angi resultat av dekningsgradsanalyse på hendelsesnivå,
eventuelt (antatte) endringer fra tidligere år.}


\chapter{Datakvalitet}\label{cha:kva}
Status og evaluering av registerets  datakvalitet er gitt under og et sammendrag
av dette fins i Tabell \ref{tab:sta}.

\section{Metoder for intern sikring av datakvalitet}\label{sec:sik}
\textit{Beskriv infrastruktur, rutiner og/eller punktaksjoner, eksempelvis
opplæring, informasjonsarbeid, tekniske støttesystemer og tilbakekobling mello
register og innregistrerende enheter.}

\section{Metode for validering av data i registeret}\label{sec:metval}
\textit{Beskriv hvilke eksterne kilder (registre) som er benyttet og hvordan
dette er sammenholdt med registerets data (design/subsampling).} 

\section{Vurdering av datakvalitet}\label{sec:valdat}
\textit{Angi viktige funn og en vurdering av resultatene fra
valideringsstudien slik som funn av systematiske feil, tilfeldige feil og
antatt effekt dette har på validitet for hele registeret.}



\chapter{Datainnsamling og formidling av resultater}\label{cha:dat}
Status og evaluering av registerets  datafangst og formidling av resultater
er gitt under og et sammendrag av dette fins i Tabell \ref{tab:sta}.

\section{Metoder for fangst av data}\label{sec:metfan}
\textit{Beskriv hva/hvem som registrerer, eksempelvis sykehusenes egne
fagsystemer, helsearbeidere, pasienter og hvordan innsamlingen skjer,
eksempelvis elektronisk eller på papirskjema.}

\section{Resultater til deltagende fagmedisinske enheter}\label{sec:resfag}
\textit{Beskriv form og frekvens,  eventuelt tilbakemeldinger på relevans,
endringsønsker mm.}

\section{Resultater til administrasjon og ledelse}\label{sec:resled}
\textit{Beskriv form og frekvens og gi en vurdering av om, og hvordan denne
informasjonen brukes av ledelse og administrasjon.}

\section{Resultater til pasienter}\label{sec:respas}
\textit{Beskriv form og frekvens og eventulet tilbakemeldinger fra pasienter,
særlig i forhold til nytten av informasjonen fra registeret.}

\section{Offentliggjøring av resultater på sykehusnivå}\label{sec:off}
\textit{Beskriv hvor/til hvem, form og frekvens.}




\chapter{Fagutvikling og kvalitetsforbedring av tjenesten}\label{cha:fag}
Status og evaluering av registerets fagutvikling og gjennomførte tiltak for
kvalitetsforbedring er gitt under og et sammendrag av dette fins i Tabell
\ref{tab:sta}.
\section{Registerets spesifikke kvalitetsmål}\label{sec:regspe}
\textit{Beskriv de mål/variabler/indikatorer som angir grad av kvalitet.
Splittes opp i resultatmessige-, strukturelle- og prosessuelle mål.}

\subsection{Resultatmål}

\subsection{Strukturmål}

\subsection{Prosessmål}

\section{Nasjonale retningslinjer utarbeidet av registeret}\label{sec:nasret}
\textit{Beskriv de nasjonale retningslinjer som er etablert av registeret.}

\section{I hvor stor grad tas retningslinjene i bruk?}\label{sec:retbru}
\textit{Beskriv i hvor stor grad de nasjonale retningslinjene er tatt i
aktiv bruk.}

\section{Deltagende enheters bruk av resultater til egen kvalitetsforbedring}\label{sec:brures}
\textit{I enheter som registrerer inn til registeret, angi konkrete prosjekter
for kvalitetsforbedring basert på resultater fra registeret. Beskriv
hvordan registeret ved styringsgruppen bidrar til oppstart og gjennomføring
av kvalitetsforbedringsarbeid lokalt hos innregistrerende enheter.}

\section{Evaluering av kvalitetsforbedrende tiltak (endret praksis)}\label{sec:evakva}
\textit{Beskriv resultater av gjennomførte tiltak for kvalitetsforbedringer.}

\chapter{Forskning og internasjonalt samarbeid}\label{cha:for}
Status og evaluering av registerets vitenskapelige aktivitet og internasjonalt
samarbeid er gitt under og et sammendrag av dette fins i Tabell \ref{tab:sta}.

\section{Samarbeid med søsterregistre eller andre fagmiljø utenfor Norge}\label{sec:samfag}
\textit{Beskriv eventuelle samarbeid registeret har med tilsvarende miljøer
utenfor Norge.}

\section{Vitenskapelige arbeider}\label{sec:vitarb}
\textit{Angi bidrag fra registeret i fagfellevurderte tidsskrifter,
vitenskapelige konferanser, utdanning av doktorgradsstipendiater eller
lignende.}


\chapter{Pasienterfaring, demografi og sosiale forhold}\label{cha:pas}
Status og evaluering av registerets bruk av pasientrapporterte resultater samt
hvordan resultater fra registereret variere med demografiske og sosiale forhold
i pasientpopulasjonen. Et sammendrag av dette fins i Tabell \ref{tab:sta}.

\section{Demografiske ulikheter}\label{sec:demuli}
\textit{Angi hvike variabler som er tilgjengelig i registeret (alder, kjønn,
bosted) og viktige funn i forhold til registerets kvalitetsmål. Vis til del
\ref{part:res} av årsrapporten.}

\section{Sosiale ulikheter}\label{sec:sosuli}
\textit{Beskriv hvilke målinger som benyttes (sivilstatus, utdanning, inntekt,
etnisitet) og viktige funn i forhold til registerets kvalitetsmål. Vis til del
\ref{part:res} i årsrapporten.}

\section{Pasientrapporterte utkommemål}\label{sec:pasutk}
\textit{Beskriv hvilke instrumenter som benyttes og gi et sammendrag av de
viktigste funn. Vis til del \ref{part:res} av årsrapporten.}











\part{Tiltak}
\textit{Basert på beskrevet status og evaluering av registeret, beskriv de
tiltak som er planlagt gjennomført fra og med neste periode.}

\chapter{Dekningsgrad}
\fyll

\section{Nye registrerende enheter}
\fyll

\section{Økning av dekningsgrad på hendelsesnivå}
\fyll

\chapter{Datakvalitet}
\fyll

\section{Endringer av intern kvalitetssikring av data}
\fyll

\section{Oppølging av resultater fra validering mot eksterne kilder}
\fyll

\chapter{Datainnsamling og formidling av resultater}
\fyll

\section{Forbedring av metoder for fangst av data}
\fyll

\section{Forbedring av resultatformidling til fagmedisinske enheter}
\fyll

\section{Forbedring av resultatformidling til administrasjon og ledelse}
\fyll

\section{Forbedring av resultatformidling til pasienter}
\fyll

\section{Forbedring av hvordan resultater på sykehusnivå offentliggjøres}
\fyll

\chapter{Fagutvikling og kvalitetsforbedring av tjenesten}
\fyll

\section{Nye kvalitetsmål}
\fyll

\section{Nye nasjonale retningslinjer}
\fyll

\section{Implementering av retningslinjer}
\fyll

\section{Økt bruk av resultater til kvalitetsforbedring i hver enkelt enhet}
\fyll

\section{Prioriterte, faglige forbedringsområder}
\fyll

\chapter{Forskning og internasjonalt samarbeid}
\fyll

\section{Nye, internasjonale samarbeidspartnere}
\fyll

\section{Forskningsprosjekter og vitenskapelig aktivitet}
\fyll

\chapter{Pasienterfaring, demografi og sosiale forhold}
\fyll

\section{Nye pasientrapporterte resultater som skal inn i registeret}
\fyll

\section{Utvidet bruk av pasientrapporterte resultater}
\fyll

\section{Nye demografiske variabler som skal inn i registeret}
\fyll

\section{Utvidet bruk av demografiske variabler}
\fyll

\section{Nye variabler som beskriver sosiale forhold hos pasienten}
\fyll

\section{Utvidet bruk av resultater som beskriver sosiale ulikheter}
\fyll



\part{Foretaksspesifikk rapportering}\label{part:spe}
\chapter{Pliktig rapportering til databehandlingsansvarlig foretak}
\textit{Her inngår de rappoteringspunkter som det databehandlingsansvarlige
foretak for registeret måtte ha som krav til sine registre utover det som
allerede er dekket i denne malen. Del \ref{part:spe} kan slettes i sin helhet
hvis slike foretaksspesifikke krav til rapportering ikke foreligger.}

\section{Første tema}
\section{Andre tema}
\section{Tredje tema}



\part{Registerfaglige resultater (tradisjonell årsrapport)}\label{part:res}

\textit{Fyll inn resultater. Relativt uavhengig ift resten av dokumentet. Dette
er ``Årsrapporten'' i tradisjonell forstand.}

%\chapter{Første tema}
%\lipsum

%\chapter{Andre tema}
%\lipsum

%\chapter{Tredje tema}
%\lipsum


%\listoffigures
\listoftables




\end{document}
