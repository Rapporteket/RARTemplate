% Mal for årsrapport for kvalitetsregistre

\documentclass{article}

% ----- Pakker ----- %

% Oppsett på dokument
\usepackage[a4paper, total={6in, 8in}]{geometry}
\usepackage[skip=10pt]{parskip}

% Språk
\usepackage[norsk]{babel}
\usepackage{csquotes}

% Figurer
\usepackage{graphicx}

% Farger på overskrifter
\usepackage{xcolor}
\usepackage{titlesec}

% Tabeller
\usepackage{tabularx}
\usepackage{longtable}
\usepackage{array}

% Symboler
\usepackage{amssymb}

% Referanser
\usepackage[style=apa]{biblatex}
\usepackage{hyperref}

% ----- Definisjoner ----- %

% Referanseliste
\addbibresource{referanser.bib}
\renewcommand{\contentsname}{Innholdsfortegnelse}

% Kulepunkter på alle nivå
\renewcommand{\labelitemii}{$\bullet$}
\renewcommand{\labelitemiii}{$\bullet$}

% Midtstilt tekst i tabellceller
\newcolumntype{P}[1]{>{\centering\arraybackslash}p{#1}}

% Farge på overskrifter a la Word
\definecolor{headline}{RGB}{46, 116, 181}
\definecolor{headline2}{RGB}{31, 77, 120}

\titleformat{\section}
{\color{headline}\huge}
{\color{headline}\thesection}{1em}{}

\titleformat{\subsection}
{\color{headline}\Large}
{\color{headline}\thesubsection}{1em}{}

\titleformat{\subsubsection}
{\color{headline2}\large}
{\color{headline2}\thesubsubsection}{1em}{}

% Sjekkbokser
\newcommand{\sjekkboks}{
    \makebox[0pt][l]{$\square$}\raisebox{.15ex}{\hspace{0.1em}$\checkmark$}
}

\newcommand{\tomboks}{
    $\Box$
}

% ----- Klar til å starte dokumentet ----- %

\begin{document}

    % Forside
    \begin{center}
        \textbf{Navn på register} \\
        \textbf{Årsrapport for [årstall]} \\
        \vspace{5mm}
        \textbf{Ola Normann\textsuperscript{1}, Medel Svensson\textsuperscript{2} og Kari Normann\textsuperscript{1}} \\
        \vspace{15mm}
        \textsuperscript{1}\textbf{et Sykehus, et Foretak, et Sted} \\
        \vspace{5mm}
        \textsuperscript{2}\textbf{et annen adresse, et annet Sted} \\
        \vspace{5mm}
        \textbf{Dato}
    \end{center}
    
    \newpage
    
    % Innholsfortegnelse
    \tableofcontents
    \newpage
    
    % Tabell over forkortelser
    \textbf{\color{headline}\large Forkortelser brukt i rapporten}
    
    \begin{center}
        \begin{tabularx}{\linewidth}{| p{4cm} | X |}
            \hline 
            Forkortelser & Forklaringer \\ \hline
            & \\ \hline
            & \\ \hline
            & \\ \hline
            & \\ \hline
            & \\ \hline
        \end{tabularx}
    \end{center}
    
    \newpage
    
    % Oppsummeringsside

    % Denne første siden skal være en oppsummeringsside med noen hovedtall fra registeret. 
    % Det er fritt opp til hvert register hvordan denne siden skal se ut. 
    % Se eventuelt andre registres årsrapporter for inspirasjon.
    % Fjern all eksempelkoden på denne siden. 
    
    Eksempler på referanser: 
    % (Navn, årstall)
    \autocite{eksempel1}
    % Navn (årstall)
    \textcite{eksempel2}
    % To referanser i samme parentes
    \autocite{eksempel1, eksempel2}

    % Lite mellomrom mellom linjer
    \vspace{5mm}
    
    Eksempel på figur: 
    
    \begin{figure}[h]
        \centering
        \includegraphics[width=0.5\linewidth]{Eksempelfigur.png}
        \caption{Figurtekst}
        \label{fig:eksempel}
    \end{figure}

    % Referere til figur
    Referanse til figur: se figur \ref{fig:eksempel}

    \vspace{5mm}
    
    Eksempel på tabell: 

    \begin{table}[h]
        \centering
        \begin{tabular}{|c|c|c|}
            \hline
            Kolonne 1 & Kolonne 2 & Kolonne 3 \\ \hline
            Verdi 1 & Verdi 2 & Verdi 3 \\ \hline
        \end{tabular}
        \caption{Tabelltekst}
        \label{tab:eksempel}
    \end{table}

    % Referere til tabell
    Referanse til tabell: se tabell \ref{tab:eksempel}
    
    \newpage
    
    % ----- Del 1 ----- %
    
    \begin{center}
        \huge
        Del 1 \\
        \vspace{5mm}
        Resultater fra registeret
        \normalsize
    \end{center}
    % La resten av siden stå tom
    \newpage
    
    
    % Sammendrag
    
    \section{Sammendrag}
    % Kortfattet sammendrag av de viktigste elementer man fra registerets side ønsker å formidle i årsrapporten. 
    % De viktigste resultater for behandlingskvalitet og kvalitetsforbedringstiltak bør angis.
    
    \subsection{Summary in English}
    % Optionally, provide a summary of the annual report. 
    % Highlights from quality assessment and improvements are relevant here. 
    
    \newpage
    
    
    % Resultater
    
    \section{Resultater}
    
    \subsection{Kvalitetsindikatorer}
    % Resultater for alle kvalitetsindikatorer. 
    % - Det skal foreligge en samletabell med navn og målnivå på alle kvalitetsindikatorene i innledningen til kapittelet.
    % - I alle figurer vises resultater på enhetsnivå der dette er mulig.
    % - Vis prosenter/gjennomsnitt og målnivå for hver enhet, i tillegg til samlet nasjonalt resultat.
    % - Dersom det er særlige hensyn som må tas ved tolkning av resultatene, anbefales registrene å beskrive dette.
    % - I tilknytning til hver kvalitetsindikator skal følgende tabell fylles ut:
    
    \begin{center}
        \begin{tabular}{|p{5cm}|p{8cm}|} \hline
        
            % Beskriv hva indikatoren viser, og eventuelt kilde (modul/skjema/PROM-data osv.)
            Definisjon/beskrivelse & \\ \hline
            
            % Angi om det er en struktur-, prosess- eller resultatindikator
            Type indikator & \\ \hline
            
            % Angi nivå for måloppnåelse
            Måloppnåelse & \\ \hline
            
            % Angi hvilke(n) nasjonal/internasjonal retningslinje(r) indikatoren er basert på, 
            % eller om indikatoren er basert på bestepraksis. Oppgi aktuelle referanser
            Kunnskapsgrunnlag & \\ \hline
            
            % Angi hvordan teller og nevner er beregnet. 
            % Eksempel: 
            % - Teller: Antall pasienter som fikk medikament x 
            % - Nevner: Antall pasienter under 80 år med hoveddiagnose xx.x
            Beregning & \\ \hline
            
        \end{tabular}
    \end{center}
    
    \subsection{Pasientrapporterte data (PROM/PREM)}
    % Beskriv hvilke instrumenter/skjema som brukes for innsamling av pasientrapporterte utfallsmål eller erfaringsmål (PROM/PREM). 
    % Beskriv utvalgte resultater for PROM/PREM som ikke inngår i kap. 2.1.
    % - I alle figurer vises resultater på enhetsnivå der det er mulig.
    
    
    \subsection{Andre analyser}
    % Resultater for øvrige analyser som ikke inngår i kap. 2.1 eller 2.2, 
    % for eksempel demografi og beskrivelse av pasientpopulasjonen. 
    % Kan med fordel deles opp i flere underkapitler.

    
    \newpage
    
    
    % ----- Del 2 -----
    
    \begin{center}
        \huge
        Del 2 \\
        \vspace{5mm}
        Administrative opplysninger
        \normalsize
    \end{center}
    % La resten av siden stå tom
    \newpage
    
    
    % Registerbeskrivelse
    
    \section{Registerbeskrivelse}
    
    % NB: Husk å oppdatere tabellen ved eventuelle endringer. 
    \begin{center}
        \begin{longtable}{| p{0.4\linewidth} | p{0.6\linewidth} |} \hline
            Bakgrunn for registeret	& 
            % Beskriv kort bakgrunnen for registeret
            
            \\ \hline
            Type register &
            % Angi om det er et diagnose-, prosedyre- eller tjenesteregister
            
            \\ \hline
            Årstall etablert &
            % Angi årstall registeret ble etablert
            
            \\ \hline
            Årstall nasjonal godkjenning &
            % Angi årstall registeret fikk nasjonal status
            
            \\ \hline
            Årstall for start av datainnsamling &
            % Angi årstall det nasjonale registeret startet datainnsamling
            
            \\ \hline
            Registerets formål &
            % Oppgi registerets formål
            
            \\ \hline
            Analyser som belyser registerets formål &
            % Beskriv kort hvilke analyser registeret gjør for å belyse formålet, 
            % f.eks. antall kvalitetsindikatorer og om resultatene publiseres på enhetsnivå for å belyse uønsket variasjon
            
            \\ \hline
            Juridisk hjemmelsgrunnlag &
            % Angi registerets juridiske hjemmelsgrunnlag
            
            \\ \hline
            Databehandler &
            % Angi registerets databehandler
            
            \\ \hline
            Databehandlingsansvarlig &
            % Angi registerets databehandlingsansvarlig
            
            \\ \hline 
            Faglig leder/ registersekretariat med kontaktinformasjon &
            % Angi registerets faglige leder og registersekretariat, med kontaktinformasjon.
            
            \\ \hline
            Fagrådets medlemmer &
            % Angi hvem som er fagrådets leder og medlemmer, med kontaktinformasjon
            
            \\ \hline
            Aktivitet i fagrådet &
            % angi aktivitet i fagrådet i rapporteringsåret, f.eks. antall møter eller om det er spesielle saker fagrådet har arbeidet med
            
            \\ \hline
            Inklusjonskriterier &
            % Angi inklusjonskriteriene for registeret, f.eks. ved diagnose-/eller prosedyrekoder
            
            \\ \hline
            Metode for datafangst &
            %  Beskriv kort hvilke data registeret samler inn, og kilde (f.eks. sykehus, pasienten). Eksempel:
            % - Hovedskjema (fra 2012)
            %   - Registreres av alle sykehus som behandler hjerneslag
            % - Oppfølgingsskjema (fra 2012)
            %   - Pasientrapporterte data innhentes 3 måneder (+/- 2 uker) etter utskrivning fra sykehuset
            % - Trombektomi-modul (fra 2020)
            %   - Registreres av sykehus som utfører trombektomi]
            
            \\ \hline
            Teknisk løsning for datafangst, og årstall for start &
            % Angi hvilken teknisk løsning registeret benytter for datafangst, og fra hvilket år, ev om registeret mangler teknisk løsning
            
            \\ \hline
            Metadata &
            % Angi om registeret har publisert sine metadata på helsedata.no, og årstall for start
            
            \\ \hline
            Innsynsløsning &
            % Angi om registeret har etablert innsynsløsning via Helsenorge, og årstall for start
            
            \\ \hline
            Antall pasienter/skjema/hendelser i rapporteringsåret &
            % Angi antall pasienter/skjema/hendelser (velg det nivået som er fornuftig for registeret) i rapporteringsåret
            
            \\ \hline
            Totalt antall pasienter/skjema/hendelser &
            % Angi antall pasienter/skjema/hendelser (velg det nivået som er fornuftig for registeret) totalt siden registeret fikk nasjonal status
            
            \\ \hline
            Stadium og nivå &
            % Angi hvilket stadium og nivå registeret er på, i hht Ekspertgruppens vurdering av fjorårets årsrapport
            
            \\ \hline
         \end{longtable}
    \end{center}
    
    \newpage
    
    \section{Datakvalitet}
    
    \subsection{Tilslutning og antall registreringer}
    % Oppgi antall individer/hendelser per enhet og nasjonalt i rapporteringsåret, i tabellformat. 
    % Enheter som skulle levert data, men som ikke gjør det, angis i tabellen med "leverer ikke data". 
    % Dersom registerets pasientgruppe ikke behandles i alle regioner, presiseres dette.
    
    
    \subsection{Dekningsgrad og responsrate}

    
    \subsubsection{Metode for beregning av dekningsgrad}
    % Angi: 
    % - ekstern kilde/register som er brukt for beregning av dekningsgrad på individnivå 
    % - hvilken periode beregningene gjelder for, samt tidspunkt for gjennomføring av dekningsgradsanalysen 
    % - hvordan analysene er gjort (metode) 
    % - dersom det ikke er hensiktsmessig å beregne dekningsgrad mot uavhengig kilde, skal registeret redegjøre for dette og beskrive hvilken annen metode som benyttes for beregning av dekningsgrad
    
    
    \subsubsection{Siste beregnede dekningsgrad}
    % Oppgi dekningsgrad i hele tall. Angi:
    % - resultat av dekningsgradsanalyse på individnivå for alle pasientgrupper/moduler samlet 
    % - dersom aktuelt: resultat av dekningsgradsanalyse på individnivå per pasientgruppe/modul
    % - dekningsgrad angis på sykehus- eller enhetsnivå

    % Dersom det er formålstjenlig, kan både tilslutning, antall individer/hendelser og dekningsgrad oppgis i én tabell, enten her eller i kap. 4.1. Henvis da til aktuelt delkapittel.]
    
    
    \subsubsection{Responsrate for pasientrapporterte data}
    % Oppgi responsrate for besvarelse av pasientrapporterte data, totalt og per sykehus/enhet hvis mulig.
    
    
    \subsection{Vurdering av datakvalitet}
    % Status og evaluering av registerets datakvalitet. 
    % For de siste 5 år: Beskriv kort hvilke metoder som er benyttet for beregning/vurdering av datakvalitet. 
    % Angi viktige funn av resultatene fra undersøkelser av variabelkompletthet, korrekthet (validitet) og reliabilitet (samsvar) for kvalitetsindikatorene, og ev. i tillegg noen av de mest sentrale variablene. 
    %
    % 1. Kompletthet: angi grad av kompletthet for sentrale variabler og variabler som inngår i kvalitetsindikatorer for rapporteringsåret.
    % 2. Korrekthet: angi grad av korrekthet for de undersøkte variablene sammenlignet med en gullstandard (f.eks. journal),
    %    og når undersøkelsen ble utført. 
    % 3. Reliabilitet, dvs. samsvar: angi grad av reliabilitet for de undersøkte variablene, og når undersøkelsen ble utført.
    % 
    % Gi en overordnet vurdering av funnene i datakvalitetsundersøkelsene og hvilken betydning dette har for tolkning av resultater fra registeret.
    % Hvis registeret har utført ytterligere validering/dokumentasjon av datakvalitet, beskrives også dette her. 
    % For nærmere beskrivelser av ulike datakvalitetsdimensjoner, se https://www.kvalitetsregistre.no/node/38.

    
    \newpage
    
    
    % Pasientrettet kvalitetsforbedring
    
    \section{Pasientrettet kvalitetsforbedring}

    
    \subsection{Identifiserte forbedringsområder}
    % Beskriv hvilke pasientrettede kvalitetsforbedringsområder som er identifisert gjennom registerets resultater/analyser for rapporteringsåret.
    % Benytt punktliste.
    
     
    \subsection{Igangsatte/utførte forbedringstiltak}
    %  For å oppfylle kravene til nivå B i stadieinndelingen må registeret både dokumentere i kap. 5.1 at det i rapporteringsåret har identifisert forbedringsområder, og i kap. 5.2 at det er igangsatt eller kontinuert pasientrettet kvalitetsforbedringsarbeid.
    %
    % For å oppfylle kravene til nivå A i stadieinndelingen skal registeret dokumentere resultater i kap 5.2 fra kvalitetsforbedrende tiltak som har vært igangsatt i løpet av de siste tre år. Tiltakene skal være basert på kunnskap fra registeret. 
    %
    % Fyll ut en tabell for hver kvalitetsindikator/forbedringsområde (kolonne A), 
    % tid tiltaket startet og evt. ble avsluttet (kolonne B), med beskrivelse av igangsatte eller kontinuerte/pågående tiltak (kolonne C), 
    % og resultater dersom dette foreligger (kolonne D). 
    % Alle tiltak og resultater tilhørende samme kvalitetsindikator/forbedringsområde beskrives i samme tabell. 
    % Resultat foreligger ofte ikke samme år som tiltaket ble startet, men gjerne først etter 2-3 år. 
    % I slike tilfeller bes det om at tabellen fylles ut også for tiltak som ble igangsatt 2 og 3 år før rapporteringsåret.
    
    
    \newpage
    
    \textbf{Tiltak og resultat} \\
    \begin{centering}
        \begin{tabularx}{\linewidth}{| X | X | X | X |}
            \hline
            \color{red}{Kolonne A: } & \color{red}{Kolonne B: } & \color{red}{Kolonne C: } & \color{red}{Kolonne D: } \\
            Aktuelt forbedringsområde & Tiltaksperiode for tiltaket & Hva ble gjort av hvem? & Hvilke resultater ble oppnådd? \\ \hline
            
            % Skriv inn tittel/navn på indikator/forbedringsområde.
            
            &
            % Oppgi når tiltaket startet, om det pågår ennå eller når det evt. ble avsluttet.
            
            &
            %  Beskriv konkrete tiltak for pasientrettet kvalitetsforbedring som er igangsatt, kontinuert/pågående eller gjennomført i rapporteringsåret. Data fra registeret må være benyttet. 
            % Gi en tydelig beskrivelse av målsettingen, hvilke konkrete tiltak som er satt inn, og hvem som har gjennomført tiltaket/ hvor det er gjennomført.
            
            &
            % Beskriv resultater av gjennomførte tiltak som har vært igangsatt i løpet av de siste tre år dersom disse foreligger.
            
            \\ \hline
                 
        \end{tabularx}
    \end{centering}
    
    \newpage

    
    % Formidling av resultater
    
    \section{Formidling av resultater}
    % Beskriv hvordan resultater fra registeret formidles til relevante mottakere. 
    % Dersom registeret ikke er i gang med slik formidling, skal registeret beskrive en konkret plan for gjennomføring av analyser og jevnlig rapportering av resultater til relevante mottakere. 
    % Punktene 1-3 er obligatorisk å fylle ut. Registeret kan i tillegg velge å fylle ut flere punkter, dersom relevant.
    
    \vspace{5mm}
    
    \begin{center}
        \begin{tabularx}{\linewidth}{| p{.5cm} | p{6cm} | p{2cm} | X |}
            \hline
            1. & 
                Årsrapport - resultatdel \vspace{\baselineskip} % La neste linje stå tom
                
                % Tekst etter denne linjen
                
                & 
                % hvor ofte?
                
                &
                % beskriv målgruppen/mottakerne for resultatene
                
                \\ \hline
                
            2. &
                Kvalitetsregistre.no \vspace{\baselineskip} % La neste linje stå tom
                
                % skriv hvor mange indikatorer som publiseres og på hvilket nivå (enhet/regional/nasjonal)
                
                &
                % hvor ofte?

                &
                %  beskriv målgruppen/mottakerne for resultatene
                
                \\ \hline
                
            3. &
                Resultater til registrerende enheter \vspace{\baselineskip} % La neste linje stå tom
                
                % Skriv hvilken form registrerende enheter kan få tilgjengeliggjort/utlevert egne aggregerte og nasjonale resultater
                
                & 
                % hvor ofte? 
                
                & 
                % beskriv målgruppen/mottakerne for resultatene
                
                \\ \hline
                
            4. &
                % Frivillig: Angi ev andre formidlingsformer, som for eksempel tilpassede rapporter til pasientgruppen, nasjonale kvalitetsindikatorer, postere/abstracts på seminarer etc. Sett inn én rad per formidlingskanal. 
                
                &
                % hvor ofte?
                
                &
                % beskriv målgruppen/mottakerne for resultatene
                
                \\ \hline
    
        \end{tabularx}
    \end{center}
    \newpage
    
    
    
    % Samarbeid og forskning
    
    \section{Samarbeid og forskning}
    
    \subsection{Samarbeid med andre fagmiljøer og helse- og kvalitetsregistre}
    % Beskriv eventuelle samarbeid registeret har med andre registre eller relevante fagmiljø, nasjonalt eller internasjonalt.
    
    
    \subsection{Datautlevering fra registrene}
    % Angi antall datautleveringer fra registeret i rapporteringsåret. 
    % Hvis ønskelig kan det angis antall utleveringer de siste tre år. 
    % Utleveringer som er en del av det daglige arbeidet i et registersekretariat skal ikke inngå i denne oversikten. 
    % Dette gjelder blant annet utlevering av data til kvalitetsregistre.no, tertialrapporter/kvartalsrapporter og rapporter ifm kvalitetsforbedringsprosjekter. Dette kan oppgis i kap. 6. 
    
    
    \vspace{5mm}
    \begin{centering}
        \begin{tabularx}{\linewidth}{| X | P{2cm} | P{2cm} | P{2cm} |}
        \hline
            Utlevering av data til følgende formål: & 2023 & 2022 & 2021 \\
            \hline
            Forskning & & & \\
            \hline
            Kvalitetsforbedring og styringsformål\footnotemark & & & \\
            \hline
            Andre formål (f.eks. til media) & & & \\
            \hline
            Totalt & & & \\
            \hline
        \end{tabularx}
        \footnotesize{\textsuperscript{1}Gjelder blant annet datautlevering etter forespørsel fra HF eller RHF, data til nasjonale indikatorer, Helseatlas o.l.}
    \end{centering}
    
    
    \subsection{Vitenskapelige artikler}
    % Angi titler på vitenskapelige artikler (publikasjonsliste) i fagfellevurderte tidsskrifter i løpet av de siste tre år, der data fra registeret inngår i datamaterialet.
    
    
    \newpage
    
    
    % ----- Del 3 -----
    
    \begin{center}
        \huge
        Del 3 \\
        \vspace{5mm}
        Stadievurdering og plan for videre utvikling av registeret
        \normalsize
    \end{center}
    % La resten av siden stå tom
    \newpage
    
    % Referanser til vurdering av stadium
    
    \section{Referanser til vurdering av stadium}
    
    \subsection{Vurderingspunkter}
    
    % Oversikt over vurderingspunkter som legges til grunn for stadieinndeling av registre med referanser til relevant informasjon gitt i årsrapporten. Denne delen fylles ut og er ment som en hjelp til registeret og ekspertgruppen i vurdering av registeret. Stadieveilederens veiledningstekst for hvert punkt beskriver utfyllende hva som skal til for at krav vurderes som oppfylt. Stadium 1 er oppfylt når registeret har status som nasjonalt. 
    
    \vspace{5mm}
    Tabell: Vurderingspunkter for \textit{[Navn på register]} og registerets egen evaluering. 

    % Erstatt "\tomboks" med "\sjekkboks" for å huke av på egen vurdering
    \begin{centering}
        \begin{longtable}{l p{0.6\textwidth} l c c}
        \hline \hline
        & & & \multicolumn{2}{l}{Egen vurdering \textit{[årstall]}} \\
        Nr & Beskrivelse & Kapittel & Ja & Nei \\
        & & & & \\
        \hline \hline
        & & & & \\
        & \textbf{Stadium 2} & & & \\
        & & & & \\
        1 & Samler data fra alle aktuelle helseregioner & 4.1 & \tomboks & \tomboks \\
        2 & Presenterer kvalitetsindikatorene på nasjonalt nivå & 2.1 & \tomboks & \tomboks \\
        3 & Har en konkret plan for gjennomføring av dekningsgradsanalyser  & 4.2 & \tomboks & \tomboks \\
        4 & Har en konkret plan for gjennomføring av analyser og jevnlig rapportering av resultater på enhetsnivå tilbake til deltakende enheter  & 6 & \tomboks & \tomboks \\
        5 & Har en oppdatert plan for videre utvikling  & 9 & \tomboks & \tomboks \\
        & & & & \\
        & \textbf{Stadium 3} & & & \\
        & & & & \\
        6 & Kan dokumentere kompletthet av kvalitetsindikatorer & 4.3 & \tomboks & \tomboks \\
        7 & Kan dokumentere dekningsgrad på minst 60 \% i løpet av siste to år & 4.2 & \tomboks & \tomboks \\
        8 & Registeret skal minimum årlig presentere kvalitetsindikatorresultater interaktivt på nettsiden kvalitetsregistre.no & 6 & \tomboks & \tomboks \\
        9 & Registrerende enheter kan få utlevert eller tilgjengeliggjort egne aggregerte og nasjonale resultater & 6 & \tomboks & \tomboks \\
        10 & Presenterer deltakende enheters etterlevelse av de viktigste faglige retningslinjer & 2.1 & \tomboks & \tomboks \\
        11 & Har en oppdatert plan for videre utvikling av registeret & 9 & \tomboks & \tomboks \\
        & & & & \\
        & \textbf{Stadium 4} & & & \\
        & & & & \\
        12 & Har i løpet av de siste 5 år dokumentert om innsamlede data er korrekte og reliable & 4.3 & \tomboks & \tomboks \\
        13 & Kan dokumentere dekningsgrad på minst 80 \% i løpet av siste to år & 4.2 & \tomboks & \tomboks \\
        14 & Presenterer minst to ganger årlig kvalitetsindikatorresultater interaktivt på nettsiden kvalitetsregistre.no & 6 & \tomboks & \tomboks \\
        15 & Registeret skal dokumentere at data anvendes vitenskapelig & 7.2, 7.3 & \tomboks & \tomboks \\
        16 & Presenterer resultater på enhetsnivå for PROM/PREM (der dette er mulig) & 2.2 & \tomboks & \tomboks \\
        & & & & \\
        & & & & \\
        & \textbf{Nivå A, B eller C} & & & \\
        & \textbf{Sett ett kryss for aktult nivå registeret oppfyller} & & \textbf{Ja} & \\
        & & & & \\
        17 & Registeret kan dokumentere resultater fra kvalitetsforbedrende tiltak som har vært igangsatt i løpet av de siste tre år. Tiltakene skal være basert på kunnskap fra registeret & 5.2 & \tomboks & \\
        & & & & \\
        & \textbf{Nivå B} & & & \\
        18 & Registeret kan dokumentere at det i rapporteringsåret har identifisert forbedringsområder, og at det er igangsatt eller kontinuert/videreført pasientrettet kvalitetsforbedringsarbeid & 5.1, 5.2 & \tomboks & \\
        & & & & \\
        & \textbf{Nivå C} & & & \\
        19 & Oppfyller ikke krav til nivå B & & \tomboks & \\
        & & & & \\
        \hline \hline
        
        \end{longtable}
    \end{centering}
    
    \newpage
    
    
    % Utvikling av registeret
    
    \section{Utvikling av registeret}
    
    \subsection{Registerets oppfølging av fjorårets vurdering fra Ekspertgruppen}
    
    % Beskriv hvordan registeret har fulgt opp ekspertgruppens kommentarer i vurderingstekst til forrige årsrapport.
    
    \subsection{Planer og behov}
    
    % Beskriv: 
    % - konkrete og realistiske tiltak for videre utvikling til neste stadium innen 3 år fra nåværende stadium ble oppnådd. 
    %   Planen bør omfatte hvert punkt som per nå ikke er oppfylt i neste stadium. 
    %   For registre som har oppnådd høyeste stadium, oppgis plan for å opprettholde stadiet, samt eventuelt videreutvikling av registeret.
    % - andre forbedringstiltak som er planlagt gjennomført for neste kalenderår
    % - behov for tekniske og andre forbedringer
    % 
    % Punktlisten under er et forslag til hva som kan inngå i en slik beskrivelse. Det kan gjerne tas med andre tiltak enn de nevnte forslagene, og punkter som ikke er aktuelle kan fjernes.
    % 
    % - Datafangst
    %   - Forbedring av metoder for fangst av data
    %     - Ressurser
    %     - Tekniske forbedringer
    %     - Organisatoriske (eksempelvis bedre organisering ved lokale sykehusavdelinger)
    % - Datakvalitet
    %   - Nye registrerende enheter/avdelinger 
    %   - Forbedring av dekningsgrad i registeret 
    %   - Forbedring av registerets kompletthet
    %   - Forbedring av rutiner for intern kvalitetssikring av data 
    %   - Oppfølging av resultater fra validering mot eksterne kilder
    % - Fagutvikling og kvalitetsforbedring av tjenesten
    %   - Nye kvalitetsindikatorer 
    %   - Nye variabler for pasientrapporterte resultater 
    %   - Utvidet bruk av pasientrapporterte resultater 
    %   - Nye demografiske variabler 
    %   - Utvidet bruk av demografiske variabler 
    %   - Bidrag til etablering av nasjonale retningslinjer eller nasjonale kvalitetsindikatorer 
    %   - Registrerende enheters etterlevelse av faglige retningslinjer 
    %   - Identifiserte kliniske forbedringsområder 
    %   - Økt bruk av resultater til pasientrettet kvalitetsforbedring i hver enkelt institusjon/enhet
    % - Formidling av resultater
    %   - Forbedring av resultatformidling til deltagende fagmiljø
    %   - Forbedring av resultatformidling til administrasjon og ledelse
    %   - Forbedring av resultatformidling til pasienter
    %   - Forbedring av hvordan resultater på institusjonsnivå publiseres
    % - Samarbeid og forskning
    %   - Nye samarbeidspartnere
    %   - Forskningsprosjekter og annen vitenskapelig aktivitet
    
    \newpage
    
    % Referanseliste
    \printbibliography[heading={bibnumbered}, title={Litteratur}]

\end{document}
